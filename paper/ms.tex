\documentclass[twocolumn]{aastex62}

% \submitjournal{ApJ}

\shortauthors{Beane et al.}
\shorttitle{Our Galactic Midplane}

\usepackage{graphicx}
\usepackage{gensymb}
\usepackage{bm}

\newcommand{\Gus}[1]{\textcolor{red}{#1}}

\newcommand{\Msun}{\ensuremath{\text{M}_\odot}}
\newcommand{\pc}{\text{pc}}
\newcommand{\kpc}{\text{kpc}}
\newcommand{\Myr}{\text{Myr}}
\newcommand{\Gyr}{\text{Gyr}}
\newcommand{\kms}{\text{km}/\text{s}}
\newcommand{\actunit}{\text{kpc}\,\kms}

\newcommand{\unit}[2]{\ensuremath{\textrm{#1}^{\mathrm{#2}}}}

\newcommand{\mi}{\texttt{m12i}}
\newcommand{\mf}{\texttt{m12f}}
\newcommand{\mm}{\texttt{m12m}}


\newcommand{\abs}[1]{\left| #1 \right|}
\newcommand{\z}{z_r}
\newcommand{\uth}{\textsuperscript{th}}
\newcommand{\n}{\text{n}}

\newcommand{\beq}{\begin{equation}}
\newcommand{\eeq}{\end{equation}}

\newcommand{\thincolor}{pink}
\newcommand{\thickcolor}{brown}
\newcommand{\halocolor}{blue}

% affiliations
\newcommand{\cca}{Center for Computational Astrophysics, Flatiron Institute,
162 5th Ave., New York, NY 10010, USA}
\newcommand{\penn}{Department of Physics \& Astronomy, University of
Pennsylvania, 209 South 33rd St., Philadelphia, PA 19104, USA}
\newcommand{\amnh}{Department of Astrophysics, American Museum of Natural
History, Central Park West at 79th St., New York, NY 10024, USA}
\newcommand{\columbia}{Department of Astronomy, Columbia University, 550 W
120th St., New York, NY 10027, USA}

\begin{document}

\title{Issues with Dynamical Measurements Across the Disk}

% \correspondingauthor{Angus Beane}
\email{abeane@sas.upenn.edu}

\author{Angus Beane}
\affil{\cca}
\affil{\penn}

\author{Robyn Sanderson}
\affil{\cca}
\affil{\penn}

\author{Mordecai-Mark Mac Low}
\affil{\cca}
\affil{\amnh}

\author{Melissa K. Ness}
\affil{\cca}
\affil{\columbia}

\author{Daniel Angl\'es-Alc\'azar}
\affil{\cca}

\begin{abstract}

Stellar actions, computed for stars from their 6D phase space measurement and
an assumed Galactic potential, are used to label and distinguish orbits. In
principle, actions have the attractive quality of being invariant and are thus
good labels of a star's orbit. Of course, inaccurate measurements of the 6D
phase space position and uncertainities in the assumed Galactic potential
induce an error in the computed action. We show that, in addition to these
complications, a systematic bias in the 6D phase space position induces a
phase-dependence in the actions, which we interpret as an error. This error is
non-Gaussian and bimodal, with neither of the modes peaking on the true action
value. A midplane error of $\sim30\,\pc$ and $\sim225\,\pc$ is necessary to
induce a $100\,\%$ error in the vertical action $J_z$ for thin and thick disk
orbits, respectively. Furthermore, we show that the local midplane varies by
$\sim100\,\pc$ at the Solar circle in Milky Way-mass cosmological zoom-in
simulations from the FIRE collaboration. Thus, current state-of-the-art action
calculations --- which assume the global and local midplanes are the same ---
include a systematic offset in $z$. Variation in the local standard of rest
will induce similar issues. The variation of the midplane must be taken into
account when performing dynamical modelling across large regions of the disk
accessible with the {\em Gaia} mission.

\end{abstract}

\keywords{actions}

\section{Introduction} \label{sec:intro}
Our understanding of the Milky Way is currently undergoing a revolution in the
era of {\em Gaia} DR2. Recent major discoveries include the remnants of a
major merger
\citep{2018ApJ...860L..11K,2018Natur.563...85H,2018arXiv180704290L,2019MNRAS.482.3426M},
a phase-space spiral \citep{2018Natur.561..360A} indicating either local
substructure infall \citep{2018MNRAS.481.1501B,2018arXiv180800451L} or bar
buckling \citep{2018arXiv181109205K}, a gap indicative of a dark matter
subtructure in GD1 \citep{2018arXiv181103631B}, among others.

These discoveries are all indicative that the Milky Way departs from
axisymmetry and is undergoing phase mixing and dynamical interactions across a
range of spatial and temporal scales. Underlying the quantitative description
of the mechanisms at work that give rise to these signatures is the assumption
of a Galactocentric coordinate system \citep{2008gady.book.....B} which can
both precisely and accurately be defined and measured. In angular coordinates,
the center of the galaxy is taken to be the location of
Sgr~A\textsuperscript{*} \citep[e.g.][]{2004ApJ...616..872R}. From stellar
motions near Sgr~A\textsuperscript{*}, the distance to the center is taken to
be $\sim 8.1\text{--}8.3\,\kpc$
\citep{2009ApJ...692.1075G,2018AA...615L..15G}. The vertical height is found
by identifying where the stellar positions and velocities reach a maximum
(effectively the median height of all disk stars), and is usually taken to be
$\sim 25\,\pc$ \citep{2001ApJ...553..184C}, with more recent measurements from
{\em Gaia} DR2 placing it at $\sim 20\,\pc$ \citep{2019MNRAS.482.1417B}. A
pre-{\em Gaia} review of these measurements is given by
\citet{2016ARAA..54..529B}.

Once a Galactocentric coordinate system has been established and a 6D phase
space measurement of a star has been made, it is desirable to convert this
measurement into action space. Actions are quantities which describe the orbit
of a star and are conserved quantities under some assumptions which we
describe next. They are the typical integral of the Hamiltonian in cylindrical
coordinates and are given by,
\beq\label{eq:actions}
J_i \equiv
\frac{1}{2\pi} \oint_{\text{orbit}}p_i\,\text{d}x_i\text{,}
\eeq
where $i=R,z,\phi$.\footnote{We typically write $J_r$ instead of $J_R$ for
aesthetic reasons.} Note that $J_{\phi} \equiv L_z$. In a slowly-evolving
axisymmetric potential, the actions of stars' orbits are conserved quantities
\citep{2008gady.book.....B,2014RvMP...86....1S}. The non-axisymmetries of a
more realistic potential induce time evolution in the {\em measured} actions
--- the true orbital invariants (if they exist) are invariant by definition.

In a slowly-evolving, axisymmetric potential, the actions of each star are
time-invariant \citep{2008gady.book.....B}. In the Milky Way, stellar actions
are expected to evolve on short time scales due to scattering with gas clouds
and on long time scales due to resonances with spiral arms, bar(s), and other
large scale perturbations \citep{2014RvMP...86....1S}. For this reason, we and
other authors have used actions to study stellar scattering in the Milky Way
disk using the improved astrometry of {\em Gaia} DR2 and various age
catalogues \citep{2018ApJ...867...31B,2018arXiv180803278T}. Actions have also
been used to study different models of spiral structure in the Milky Way
\citep{2019MNRAS.tmp..155S}. A summary of the disk in action space is given by
\citet{2018arXiv180503653T}.

The transformation to action space is not trivial, and many of the assumptions
of the transformation are potentially incorrect. For instance, the assumption
that the Galactic potential is axisymmetric may be inadequate with current
measurement errors. Even if this assumption is valid, the parameters used in
our axisymmetric models may be inadequately tuned to observations. In
addition, the necessity of 6D phase space measurements of a star limits the
Galactic volume observable in action space.

With these limitations in mind, it is therefore natural to consider the reason
to transform to action space. One reason is dimensionality reduction --- in
the action space paradigm, a stellar orbit is completely described by three
actions, as opposed to six dimensions of phase space. This means that the
relationship between {\em orbital} properties of individual stars and stellar
properties such as age or metallicity can be investigated.

We explore one science goal that relies on the conservation of actions --- the
reconstruction of open clusters across large distances using dynamical
tagging. The basic program is to match stars with roughly the same actions as
their supposed natal cluster. To set the scene, we consider the orbits of the
nine open clusters within $250\,\pc$ as reported by the {\em Gaia}
collaboration \citep{2018AA...616A..10G}. In Table~\ref{tab:real_clusters}, we
report the distance to each cluster, the three actions, and the maximum
vertical extent of their orbits ($z_{\text{max}}$). The actions are computed
using the technique described in Section~\ref{ssec:action_comp}. We also show
the mass of each cluster and the ``neighborhood'' we compute for each cluster,
which gives a rough sense of how far one can move away from the cluster before
the issue which we discuss next becomes important. This neighborhood is
discussed in more detail in Section~\ref{sssec:reconstruction}.

\begin{deluxetable*}{cCCCCCCC}
\tablecaption{Relevant dynamical parameters for the $9$ open clusters within
$250\,\pc$ of the Sun, using the relevant initial positions and velocities
from \citet{2018AA...616A..10G}. The neighborhood column is explained in more
detail in Section~\ref{sssec:reconstruction}.\label{tab:real_clusters}}
\tablehead{\colhead{cluster} & \colhead{distance} & \colhead{$J_r$} &
\colhead{$L_z$} & \colhead{$J_z$} & \colhead{$z_{\text{max}}$} & \colhead{mass} & \colhead{neighborhood} \\ 
\colhead{ } &
\colhead{$\mathrm{pc}$} & \colhead{$\mathrm{kpc\,km\,s^{-1}}$} &
\colhead{$\mathrm{kpc\,km\,s^{-1}}$} & \colhead{$\mathrm{kpc\,km\,s^{-1}}$} &
\colhead{$\mathrm{pc}$} & \colhead{$M_{\odot}$} & \colhead{$\mathrm{pc}$}}
\startdata
alphaPer & 174.9 & 14.0 & -1739 & 0.0045 & 10.5 & 352\tablenotemark{b} & 57 \\
Blanco1 & 237.2 & 1.6 & -1841 & 1.53 & 207.0 & 410\tablenotemark{c} & 1460 \\
ComaBer & 85.9 & 1.6 & -1860 & 0.71 & 139.7 & 112\tablenotemark{d} & 522\\
Hyades & 47.5 & 20.3 & -1757 & 0.26 & 81.5 & 400\tablenotemark{e} & 462 \\
IC2391 & 151.6 & 5.5 & -1790 & 0.018 & 21.2 & & \\
IC2602 & 152.2 & 11.5 & -1718 & 0.32 & 88.6 & & \\
NGC2451A\tablenotemark{a} & 193.7 & 5.8 & -1805 & 0.21 & 73.4 & & \\
Pleiades & 135.8 & 19.3 & -1696 & 0.38 & 96.3 & 800\tablenotemark{f} & 685 \\
Praesepe & 186.2 & 21.2 & -1766 & 0.58 & 122.9 & 550\tablenotemark{d} & 781
\enddata

\tablenotetext{a}{This cluster is labeled NGC2451, which we interpret as
NGC2451A since NGC2451B lies further than $250\,\pc$ away.}
\tablenotetext{b}{\citet{2016MNRAS.457.1028S}}
\tablenotetext{c}{\citet{2007AA...471..499M}}
\tablenotetext{d}{\citet{2007AJ....134.2340K}}
\tablenotetext{e}{\citet{1998AA...331...81P}}
\tablenotetext{f}{\citet{2001AJ....121.2053A}}

\tablerefs{\citet{2018AA...616A..10G}}
\end{deluxetable*}

We propose a new type of error in the computation of actions which has
implications for general dynamical modelling across the disk. Our main point
is that the ``local'' Galactic midplane is not the same as the ``global''
Galactic midplane. If we observe a group of stars that are
non-local\footnote{i.e. a group of stars that is far enough away. The exact
distance depends on the accuracy of the measurement being made and the goals
of the particular problem.}, they will know of a different local midplane than
our local midplane extrapolated onto their position. This means that when we
convert those star's positions and velocities from a heliocentric to a
Galactocentric coordinate system, we are inducing a systematic bias in those
stars' $z$ values. The same applies to inaccuracies in measurements of
$R_{\odot}$ and variations in the local standard of rest (LSR).

We do not propose what a modeller should use for the ``true'' global midplane,
since such a definition probably depends on the particular problem such a
modeller is facing. Even without a high-precision definition and measurement
of the global Galactic midplane, we can still assess how different we should
expect the local midplane to be by measuring the azimuthal dependence of the
local midplane. This procedure may be readily tractable over some of the disk
with current datasets, using appropriate selection functions. However, we turn
to simulation in this work because we can investigate the entire disk.
Specifically, we use cosmological zoom-ins of Milky Way-mass galaxies from the
Feedback in Realistic Environments (FIRE)
collaboration\footnote{\url{https://fire.northwestern.edu}} (the {\em Latte}
suite of simulations).

Our main point is that the local midplane varies between different points in
the Galaxy. This is a different statement than that our measurements of the
local midplane are erroneous. Indeed, the local midplane has recently been
measured to a claimed high level of precision, setting $z_{\odot} = 20.8 \pm
0.3\,\pc$ \citep{2019MNRAS.482.1417B}. However, we also do not mean to say
that determinations of the local midplane are perfect, since midplane
measurements from H~\textsc{ii} regions place $z_{\odot} \sim 5-10\,\pc$. An
additional complication is the fact that non-zero Solar heights induce a tilt
in the Galactic midplane \citep{2014ApJ...797...53G,2016ARAA..54..529B}.

In Section~\ref{sec:ref_frame}, we describe the general impact coordinate
system errors have on the measured actions. In Section~\ref{sec:local_fire},
we measure the local midplane in cosmological zoom-in simulations from the
FIRE collaboration. In Section~\ref{sec:discussion} we discuss and give some
context for our results, and implications for the definition of the Solar
neighborhood. We conclude in Section~\ref{sec:conclusion}.

\section{Motivation} \label{sec:ref_frame}
We first demonstrate the significance to action computations, especially for
disk-like orbits, of systematic errors in the determination of the Galactic
midplane and distance to the Galactic center. The consequences we explore here
may also arise from various other systematic errors. For instance, the
axisymmetric Galactic potential model used in many works to compute actions
may not be a good description of the true potential --- or the parameters used
may yield a potential that is systematically incorrect outside an original
fitted region. In this section, we assume that the Galaxy is perfectly
described by our model axisymmetric potential, and simply explore the
consequences of systematic errors in the determination of the Galactocentric
coordinate system.

\subsection{The effect of midplane determination on orbit and action
calculations} \label{ssec:cartoon} We present a cartoon in
Figure~\ref{fig:cartoon} to show how an inaccurate determination of the
midplane leads to a phase-dependence of the actions calculated from an orbit
specified by a phase-space point and an assumed potential model. The $y$-axis
corresponds to the vertical height of the orbit and the $x$-axis orbital
phase. The solid gray line indicates the ``true'' orbit of the star; that is,
its orbit as it oscillates around the true midplane. The dashed gray line is
an erroneous midplane assigned to the potential model which an observer uses
to integrate the orbit of the star and hence estimate its actions. The model
potential is otherwise identical to the one in which the star is actually
moving.

Now suppose this observer makes a measurement of the true orbit at the blue
point or the pink point (i.e. at two different orbital phases). Then the blue
and pink lines correspond to the orbits that the observer would compute for
each point based on the potential model with the erroneous midplane. In action
space, this would correspond to a different value of $J_z$ for the blue and
pink points. In this way, assuming the wrong coordinate system induces a
phase-dependence in the actions estimated for the star, which in the correct
potential (in this example, the one with the correct midplane) should be
phase-independent.

\begin{figure*}
\plotone{fig/cartoon.pdf}
\caption{Cartoon showing the effect an error in the determination of the
coordinate midplane can have on orbit integration and action estimation. The
$x$-axis shows the orbital phase and the $y$-axis the vertical height. The top
gray curve depicts an example ``true'' orbit oscillating about the true
midplane (horizontal solid gray line). Consider an observer who erroneously
assumes the midplane is located at the horizontal dashed line. Suppose this
observer measures the phase-space position of the star at two different
orbital phases (blue and pink points). If the observer were then to integrate
the star's orbit using a model potential with the erroneous midplane, they
would obtain the blue and pink curves for the star's orbit, respectively. The
actions estimated from these two  erroneous orbits would subsequently differ,
both from each other and from the actions estimated for the true orbit (in the
potential with the correct midplane). Hence an incorrect midplane in the
potential model assumed will induce phase-dependence in the actions estimated
for a given star in that potential.}
\label{fig:cartoon}
\end{figure*}

\subsection{Numerical methods} \label{ssec:action_comp}
We compute actions as in our previous work \citep{2018ApJ...867...31B}, using
the code \texttt{gala} v0.3 to perform orbit integrations and conversion to
action space \citep{2017JOSS....2..388P,Price-Whelan:2018}. To compute actions
we use the torus-mapping technique first presented by
\citet{1990MNRAS.244..634M} and adapted by \citet{2014MNRAS.441.3284S} to
calculate actions for an orbital time-series starting from a phase-space
position $(x, v)$ and integrated in a potential $\Phi$. We use the default
\texttt{MWPotential} as our potential, which is based on the Milky Way
potential available in \texttt{galpy} \citep{2015ApJS..216...29B}. This
potential includes a Hernquist bulge and nucleus \citep{1990ApJ...356..359H},
a Miyamoto-Nagai disk \citep{1975PASJ...27..533M}, and an NFW halo
\citep{1997ApJ...490..493N}, and is fit to empirically match some
observations. We use the Dormand-Prince 8(5,3) integration scheme
\citep{Dormand80:integrator} with a timestep of $1\,\Myr$ and integrate for
$5\,\Gyr$, corresponding to $\sim 20$ orbits for a Sun-like star.

Other methods for computing actions are used in the literature. For example,
the St\"ackel Fudge method \citep{2016MNRAS.457.2107S}, which uses a single
St\"ackel potential (with analytic actions) to approximate the Galactic
potential \citep{1985MNRAS.216..273D,2012MNRAS.426.1324B}, was used in many
recent works exploring actions in the Galactic disk
\citep[e.g.][]{2018arXiv180503653T,2018MNRAS.481.4093S,2018arXiv180803278T}.
For disk-like orbits, existing implementations of the St\"ackel Fudge method
are of acceptable accuracy, but since we also consider halo-like orbits in
this work we choose to use orbit integration and torus mapping throughout
\citep{2016MNRAS.457.2107S}.

\subsection{Quantification of the effect of systematic coordinate-system
errors} \label{ssec:quant}
First, we illustrate the phase-dependence induced by an
example systematic error in the Galactocentric coordinate system. We will
consider three orbits in the model potential described in
Section~\ref{ssec:action_comp} that are typical of stars in the stellar thin
disk, stellar thick disk, and the stellar halo. We summarize their initial
positions in phase space and the actions computed by integrating their orbits
in the correct potential in Table~\ref{tab:orbits}. Each orbit, integrated
without systematic coordinate errors, is plotted in Appendix~\ref{app:orbits}.
We will refer to these orbits by their names (thin-disk, thick-disk, halo)
henceforth.

We begin by considering the thick-disk orbit. Suppose that we could observe
the thick-disk star's phase-space position at many different times (and hence
different orbital phases), but used a coordinate system in which the midplane
is systematically offset in height by $100\,\pc$ from its actual location ---
i.e. we subtract the vector $(0, 0, 100)\,\pc$ from each position in the
orbit. This corresponds to an observer physically located at the position $(8,
0, 0)\,\kpc$ in the coordinate system of the true potential, but erroneously
thinking they are located at $(8, 0, 0.1)\,\kpc$.

\begin{deluxetable*}{ccccccc}
\tablecaption{Description and names of the three orbits considered in this
work.\label{tab:orbits}}
\tablehead{\colhead{name} & \colhead{initial position} & \colhead{initial
velocity} & \colhead{$J_r$} &
\colhead{$L_z$} & \colhead{$J_z$} & \colhead{$z_{\text{max}}$}\\ \colhead{ } &
\colhead{$\mathrm{kpc}$} & \colhead{$\mathrm{km}/\mathrm{s}$} &
\colhead{$\mathrm{kpc\,km\,s^{-1}}$} &
\colhead{$\mathrm{kpc\,km\,s^{-1}}$} & \colhead{$\mathrm{kpc\,km\,s^{-1}}$} &
\colhead{$\mathrm{kpc}$}}
\startdata 
thin-disk & $(8, 0, 0)$ & $(0, -190, 10)$ & 40.3 & -1520 & 0.69 & 0.12 \\
thick-disk & $(8, 0, 0)$ & $(0, -190, 50)$ & 32.5 & -1520 & 23.0 & 0.85 \\ 
halo & $(8, 0, 0)$ & $(0, -190, 190)$ & 32.8 & -1520 & 529.1 & 6.16
\enddata
\end{deluxetable*}

Starting from each observation of the star's phase-space position that such an
(extremely long-lived) observer makes every $\Myr$, we then compute the
actions as outlined above using an integration from that starting point in
phase space carried out using the true potential (but specifying the star's
present-day position using the systematically offset coordinate system). The
erroneous computed actions for each phase-space starting point are shown for
the first $\Gyr$ of the orbit in the {\em upper} panels
Figure~\ref{fig:one_orbit_wrong_ref}.\footnote{Occasionally the numerical
scheme fails and very large actions are reported by \texttt{gala}---we perform
a $5\sigma$ clip on each action to exclude such orbits, but this only excludes
a total of $5$ orbits out of the $1000$ considered for
Figure~\ref{fig:one_orbit_wrong_ref}. Some numerical artifacts remain, but the
vast majority of orbits are computed properly.}
    
We also perform the same procedure but assume a $100\,\pc$ offset in the $x$
    component (i.e. subtracting the vector $(100, 0, 0)\,\pc$) in the {\em
    lower} panels. This is equivalent to a systematic offset in the
    determination of the distance from the Sun to the Galactic center.

Figure~\ref{fig:one_orbit_wrong_ref} shows that the actions computed in the
offset coordinate systems depend on the time (i.e. orbital phase) at which the
star's phase-space position is observed. (To our extremely long-lived observer
this would appear as time-dependence in the actions, which appear to oscillate
around their value using the correct coordinate system even though in this
example the Galaxy is described perfectly by a static, axisymmetric
potential.) The relative size of the phase variation in each action depends on
the direction of the systematic offset as well as the true values of the
actions (i.e. the type of orbit). In reality we will have one measurement of
the phase-space position to work with, in which case the determination of the
orbital phase in $z$ or $R$ is degenerate with the degree of systematic offset
in that coordinate (see Figure~\ref{fig:cartoon}). Additionally, in searching
for migrated members of known open clusters, we would compute actions for
stars with a wide range of orbital phases and locations in the Galaxy. In the
following we therefore quote percentile ranges for the possible values
computed for each action as a proxy for the effect of these systematic errors
in the coordinate system.

For a systematic offset in $z$ ({\em upper} panels), the
95\textsuperscript{th} minus 5\textsuperscript{th} percentiles are $2.2$ and
$6.2 \actunit$ for $J_r$ and $J_z$, respectively. These are fractional errors
of $5.7\%$ and $85.7\%$, respectively. The error induced in $L_z$ is
negligible, as expected since $L_z$ is independent of $z$. It is worth
pointing out that a $100\,\pc$ error in an orbit with
$z_{\text{max}}=850\,\pc$ --- a $12\%$ error
--- induced a $43\%$ error in the computation of $J_z$.

For a systematic offset in $x$ (or Galactic center distance), the
95\textsuperscript{th} minus 5\textsuperscript{th} percentiles are $6.9$,
$47$, and $0.71 \actunit$ for $J_r$, $L_z$ and $J_z$, respectively. These are
fractional errors of $21\%$, $3.1\%$ and $3.1\%$, respectively.

\begin{figure*}
\plotone{fig/schmactions_one_orbit.pdf}
\caption{The artificial phase-dependence in the observed actions induced by an
error in the Galactocentric coordinate system. We consider here the thick-disk
orbit, which has actions of $(J_r, L_z, J_z) = (37.9, -1520, 7.0)\,\actunit$
and $z_{\text{max}}=850\,\pc$ (see Table~\ref{tab:orbits}). We integrate the
orbit according to the procedure laid out in Section~\ref{ssec:action_comp}.
Then, we subtract $100\,\pc$ from the $z$ value ({\em upper panels}) or the
$x$ value ({\em lower panels}) of each position in the orbit, corresponding to
an erroneous observer assuming a midplane ({\em upper}) or solar radius ({\em
lower}) that is off by $100\,\pc$. We then allow our (immortal) observer to
observe the orbit over $1\,\Gyr$ and perform the same orbit integration
procedure at each timestep, and report the values of the actions. The
computation of $L_z$ is pristine to errors in $z$, with only numerical
artifacts remaining. Only small errors are induced in $J_r$, with the middle
$90\%$ of values over the $\Gyr$ being within $\sim8\%$ of the true $J_r$. As
expected, large errors are induced in $J_z$ with a $100\,\pc$ offset in $z$,
with the middle $90\%$ of values being within $\sim43\%$ of the true $J_z$.
The $x$ offset induces uncertainties in $J_r$ and $L_z$ of $\sim21\%$ and
$\sim3\%$, respectively. A $\sim3\%$ error in $J_z$ is also induced.}
\label{fig:one_orbit_wrong_ref}
\end{figure*}

We now repeat the same procedure for systematic offsets between $0$ and
$500\,\pc$ in the $z$ and $x$ components. In
Figure~\ref{fig:many_orbit_wrong_ref}, we report the 95\uth minus 5\uth
percentile divided by the true action value for the three different fiducial
orbits in Table \ref{tab:orbits}. The thick-disk (\thickcolor) orbit is the
one from Figure~\ref{fig:one_orbit_wrong_ref}, but we also now consider the
effect on the action determined for the thin-disk (\thincolor) and halo
(\halocolor) orbits.

The top row of Figure~\ref{fig:many_orbit_wrong_ref} shows the spread induced
in each action for an offset in the $z$ component (i.e. the Galactic
midplane). In the {\em bottom row} we consider offsets in the $x$ component
(i.e. the Solar radius). The {\em left}, {\em middle}, and {\em right} columns
show the fractional spread in the values computed for $J_z$, $L_z$, and $J_r$,
respectively. We compute the fractional spread $\Delta J_i/J_i$ as the 95\uth
minus 5\uth percentile of the action computed using the offset coordinate
system over the course of the first $\Gyr$, divided by the true action
(computed in the correct coordinate system).

In the {\em upper middle} panel, there is essentially no spread in the
determination of $L_z$. This is expected since, in principle, $L_z$ only
depends on the $x$- and $y$-components of the position and velocity of the
stars\footnote{In practice, however, $L_z$ is computed as part of the
torus-fitting method.}, and is thus unaffected by offsets in $z$. Indeed, the
result we found in Figure~\ref{fig:one_orbit_wrong_ref} for the thick-disk
orbit holds for all orbit types.
 
The {\em upper right} panel shows that the fractional error in $J_z$ is more
exaggerated for more planar (disk-like) orbits. For the thin-disk orbit, a
systematic offset of $\sim30\,\pc$ in the $z$ coordinate already results in
$100\%$ deviations in the actions. The required offset for $100\%$ deviation
is $\sim225\,\pc$ for the thick-disk orbit. The halo orbit is relatively
resistant to errors in the midplane, with only $\sim25\%$ error in $J_z$ out
to a midplane offset of $500\,\pc$.

For the offset in the Solar radius {\em bottom panels}, the error is largest
for $J_r$, with some deviations resulting in $L_z$ and relatively small
deviations in $J_z$. In the {\em bottom middle} and {\em bottom right} panels
all three lines nearly overlap.

\begin{figure*}
\plotone{fig/schmactions_many_orbits.pdf}
\caption{We report the 95\uth minus 5\uth percentile of the error in the
measured action ($\Delta J_i$) from coordinate system errors for the thin,
thick, and halo orbits (Table~\ref{tab:orbits}). The {\em left}, {\em center},
and {\em right} panels show the result for $J_r$, $L_z$, and $J_z$,
respectively. The {\em upper} panels consider an offset in $z$ and the {\em
lower} panels consider an offset in $x$ (equivalently, an offset in the Solar
radius).}
\label{fig:many_orbit_wrong_ref}
\end{figure*}

\begin{figure}
\plotone{fig/schmactions_Jz_hist.pdf}
\caption{A histogram of the observed values in $J_z$ for the thick-disk orbit
assuming a $z$ offset of $100\,\pc$. One can see that if the observed $z$
values have a bias (from e.g. an incorrectly computed midplane), then the
induced error distribution in $J_z$ is decidedly non-Gaussian. Therefore, any
sort of error propagation must take this into account.}
\label{fig:Jz_hist}
\end{figure}

In Figure~\ref{fig:Jz_hist} we plot a histogram of the observed values of
$J_z$ for the thick-disk orbit assuming a $z$ offset of $100\,\pc$ (i.e. the
{\em right} panel of Figure~\ref{fig:one_orbit_wrong_ref}). The true value of
$J_z$ is plotted as a vertical dashed line. Here we see that the spread in
$J_z$ induced by a systematic offset in $z$ is non-Gaussian and bimodal.
Furthermore, neither of the modes lie at the true value of $J_z$.

\section{Azimuthal midplane variations at the Solar Circle: estimates from
FIRE} \label{sec:local_fire}
\subsection{Description of Simulation} \label{ssec:cosmozoom}
We briefly describe the cosmological-hydrodynamical zoom-in simulations used
in this work. Cosmological zoom-ins allow one to simulate a selected region at
high resolution embedded in a low-resolution cosmological background
\citep[e.g.][]{1993ApJ...412..455K,2014MNRAS.437.1894O}, to which can be added
a hydrodynamical simulation of the baryonic component with recipes for star
formation and feedback. The FIRE collaboration has simulated a number of Milky
Way-mass galaxies using this technique as part of the {\em Latte} suite of
FIRE-2 simulations \citep{2016ApJ...827L..23W,2018MNRAS.481.4133G}. For this
work we focus on the three zoom-ins considered in \citet{2018arXiv180610564S},
which show broad agreement of many of their global properties with
observations of the Milky Way. The $\z=0$ snapshots\footnote{In this work, to
avoid confusion with the vertical height $z$, we refer to cosmological
redshift as $\z$.} of these three simulations, called \mi{}, \mf{}, and \mm{}
for shorthand, are publicly available alongside associated mock {\em Gaia} DR2
catalogues generated from them.\footnote{at \url{http://ananke.hub.yt}}

These simulations contain dark matter particles of mass $\sim35,000\,\Msun$,
gas particles of mass $\sim 7000$ to $20,000\,\Msun$, and star particles
of mass $\sim 5000 \text{--} 7000\,
\Msun$,\footnote{In fact, the simulation \mi{} contained a gas
splitting bug which causes a small number of gas and star particles to have
larger than typical masses. These constitute only $\sim0.2\%$ of the star
particles in the $\z=0$ snapshots, and are only a factor of a few more massive
than typical star particles. Since our results are broadly consistent between
the three simulations, we ignore this minor complication.} with the lower end
coming from stellar evolution \citep{2018arXiv180610564S}. Softening lengths
for dark matter and star particles are fixed at $112\,\pc$ and $11.2\,\pc$,
respectively.\footnote{This is $2.8$ times the often-quoted
Plummer-equivalent.} The gas softening length is adaptive, but at $z=0$ the
median softening length for cold ($T < 10^4\,\text{K}$) gas particles around
roughly solar positions (with cylindrical radii within $500\,\pc$ of
$8.2\,\kpc$ and $\abs{z}<1\,\kpc$) is $12.4, 12.5, 10.3\,\pc$ for \mi{},
\mf{}, and \mm{}, respectively. These values are summarized in
Table~\ref{tab:scale_height}.


\begin{deluxetable*}{cccccc}
\tablecaption{Stellar disk scale heights of the real Milky Way and the
simulated galaxies considered in this work.\label{tab:scale_height}}
\tablehead{\colhead{galaxy} & \colhead{cold\tablenotemark{a} gas disk} & \colhead{stellar thin disk} & \colhead{stellar thick disk} & \colhead{cold gas softening length} & \colhead{stellar softening length} \\ 
\colhead{} & \colhead{(pc)} & \colhead{(pc)} & \colhead{(pc)} & \colhead{(pc)} & \colhead{(pc)} } 
\startdata
MW\tablenotemark{b} & 40 & 300 & 900 & 0 & 0 \\
\mi\tablenotemark{c} & 800\tablenotemark{d} & 480 & 2000 & 12.4 & 11.2 \\
\mf\tablenotemark{c} & 360 & 440 & 1280 & 12.5 & 11.2 \\
\mm\tablenotemark{c} & 250 & 290 & 1030 &10.3 & 11.2 \\
\enddata
% \tablerefs{\citet{2016ARAA..54..529B,2018arXiv180610564S,2008ApJ...673..864J}}
\tablenotetext{a}{$T<1000$ K}
\tablenotetext{b}{\citet{2008ApJ...673..864J,2016ARAA..54..529B}}
\tablenotetext{c}{\citet{2018arXiv180610564S}}
\tablenotetext{d}{The azimuthally averaged gas vertical density profile in
m12i is nearly constant to this height, though individual regions show smaller
scale heights and dense clouds.}
\end{deluxetable*}


The softening lengths used in the simulations can affect the ability to
resolve the very thinnest planar structures, which in turn can affect how much
the density-based midplane varies as a function of azimuth. The Milky Way's
dense, star-forming gas disk is thought to have a scale height of about $40\,\pc$
($3\text{-}4$ times the cold gas softening length; \citealt{2019ApJ...871..145A}) and
the thin stellar disk a scale height of about 300 pc (30 times the stellar
softening length; \citealt{2008ApJ...673..864J}). We therefore expect that
resolution effects may still be affecting the scale heights of these
components in the simulations, especially the cold gas. Indeed, the stellar
scale heights of the simulated galaxies are equal to or larger than the MW's
while the gas scale heights are significantly larger (although the proper
basis comparison is less clear in the case of the gas; the quoted value for
the Milky Way comes from studies of high-mass star-forming regions). As noted
above, the midplanes defined by gas and stars can be tilted with respect to
one another as well, precluding extending the precision of the gas midplane
definition to the stellar component. In this work we specifically consider
azimuthal variations in the midplane \emph{as defined by the stellar mass
density}, which will subsequently be lower in the simulations than in the MW:
within 200 pc of the Sun the measured gas and stellar volume densities in the
MW are roughly comparable at $\sim .04\ \Msun\
\unit{pc}{-3}$, while in the simulations they are a factor of a few lower (see
Table 3 of \citealt{2018arXiv180610564S}). We thus expect that the degree of
midplane variation we find in the simulated systems is likely to be
\emph{less} than in the real MW, since on the smallest scales the simulations
are \emph{smoother} than the real Galaxy.


Cosmological simulations of Milky Way-mass galaxies are not perfect
representations of the true Milky Way in other ways as well, as discussed in
\citet{2018arXiv180610564S}. For instance, the velocity structure of \mi{} is
closer to M31's than the Milky Way's (Loebman et al. in prep). However, in
this work we are most interested in the global properties of the potential,
and specifically in deviations from axisymmetry. From this perspective, the
simulated galaxies are actually \emph{more} axisymmetric than we might expect
of the Milky Way. While they have prominent spiral arms, none has as strong a
bar as the MW does at present day, and none has a nearby companion like the
Large Magellanic Cloud. One of the three we consider (\mf) does have an
ongoing interaction with a satellite galaxy similar to Sagittarius, which has
punched through the galactic disk outside the solar circle and induced some
warping. However, we expect that, as with the use of finite smoothing lengths,
the degree of midplane variation in the simulations is likely a factor of a
few lower than what could be expected in the MW.

In this work, we take the galactocentric coordinate system described in
Section 3 of \citet{2018arXiv180610564S} as our fiducial coordinate system for
each galaxy. In short, the center of the galaxy is found iteratively, the
center of mass velocity is then determined by all star particles within
$15\,\kpc$ of this center, and the galaxy is then rotated onto a
principal-axis frame determined by stars younger than $1\,\Gyr$ inside of the
fiducial solar radius $R_{\odot} = 8.2\,\kpc$, such that the disk plane is the
$x$--$y$ plane.

\subsection{Local Midplane} \label{ssec:local_midplane}
Using the three simulations, we determine the local midplane that an observer
might measure if they were situated in each of these galaxies, as a function
of azimuth at the solar circle. Starting from the coordinate system described
in the previous section, which is approximately aligned so that the $z$
coordinate is perpendicular to the disk plane, we place our imaginary observer
at $z=0$ and a solar radius of $8.2\,\kpc$ and vary the azimuth between
$0<\phi<2\pi$\footnote{Our very long-lived observer also has warp-drive.}. At
each value of $\phi$ we then compute the median $z$ for stars within a
cylinder of radius $0.5\,\kpc$ and height $1\,\kpc$ perpendicular to the
fiducial disk. We then re-define this median $z$ as the new midplane of the
cylinder, re-select stars, and iterate until the median $z$ value converges.
We find that only $10$ iterations of this procedure are necessary for
convergence. The resulting median $z$ is taken to be what our observer would
measure as the local Galactic midplane at each $\phi$. This procedure assumes
perfect density estimation, and therefore perfect corrections for extinction,
within the cylinder defining the ``solar neighborhood.'' Imperfect extinction
correction is likely to increase the amplitude of the estimated fluctuations
in $z$. To account for the effect of particle noise, we bootstrap-resample
stars within a cylinder of stars with height $2\,\kpc$ and the same radius
$1000$ times and determine the $95^{\text{th}}$ minus $5^{\text{th}}$
percentile range. This bootstrap resampling is performed by first selecting
all stars within a height of $2\,\kpc$, resampling that selection, and then
repeating the $10$ iterations.

To allow for potential small inaccuracies in the determination of the original
fiducial coordinate system, we also subtract the best fit $\text{A}
\sin{\left(\phi + \text{B}\right)} + \text{C}$ curve from the midplane as a
function of azimuth to account for an overall tilt of the midplane (a
simplified version of the strategy described in
\citealt{2019ApJ...871..145A}). For $\text{A}$ the values are $-174$, $62$,
and $3.8\,\pc$, for $\text{B}$ the values are $0.3$, $0.65$, and
$0.021\,\text{rad}$ and for $\text{C}$ the values are $-47$, $18$, and
$-15\,\pc$ for \mi{}, \mf{}, and \mm{}, respectively. For the assumed solar
radius of $8.2\,\kpc$, we can approximate the angle offset for the $z$-axis
from the values of $\text{A}$ --- we compute $1.22$, $0.43$, and $0.03\,\deg$
for \mi{}, \mf{}, and \mm{}. These angle offsets are consistent with the
values given in \citet{2018arXiv180610564S} for the difference between the
$z$-axis as defined by the gas and stars.

Figure~\ref{fig:midplane} shows the relative $z$ location of the inferred
midplane our imaginary observer would determine as a function of azimuth for
each galaxy, using their local ``solar neighborhood'' (the cylinder defined
above). The $5\textup{--}95$ percentile range from the bootstrap procedure is
shown as the dashed-line error bars. The $90\%$ interquartile range for each
galaxy is $185$, $162$, $84\,\pc$ for \mi{}, \mf{}, and \mm{}. In two of the
three cases the midplane therefore varies by more than $\pm 100\,\pc$
depending on the azimuth along the Solar circle; in the third (\mm{}, which
has the thinnest ``thin disk'' of stars, but the largest stellar mass) the
variation is more like $\pm 50\,\pc$.

\begin{figure*}
\plotone{fig/midplane_fit.pdf}
\caption{The local midplane determined at the fiducial Solar radius
($8.2\,\kpc$) for the three FIRE galaxies \mi{}, \mf{}, and \mm{} ({\em left},
{\em center}, and {\em right} panels). The local midplane is determined at a
position $\phi$ by taking the median height of all stars within $R=0.5\,\kpc$
and $z=1\,\kpc$ (in cylindrical coordinates). The procedure is performed again
using the new height $10$ times to converge on the local midplane height. In
order to allow for the possibility that the fiducial Galactocentric coordinate
system is incorrect, we subtract the best fit $A\sin{(\phi+B)}+C$ curve from
each panel --- this figure is reproduced with the original midplane
determination (i.e. before subtracting the best fit sine curve) in
Appendix~\ref{app:lsr}. We then bootstrap resample $1000$ times on all stars
within a $2\,\kpc$ height of the fiducial midplane to determine error bars
(95\uth and 5\uth percentiles), which we report as dashed lines.}
\label{fig:midplane}
\end{figure*}

\subsection{LSR Variations} \label{ssec:lsr_var}
We also expect that the local standard of rest (LSR) should vary as a function
of azimuth. We perform this calculation in Appendix~\ref{app:lsr} to estimate
, but performing a best-fit subtraction to correct for misalignment of the
original coordinate system (as in the previous section) is more involved.
Since the variation in the LSR is less pronounced than for the midplane, and
since the dominant effect is limited to $J_r$
(Figure~\ref{fig:many_orbit_wrong_ref}), we defer this calculation to future
work.


\section{Discussion} \label{sec:discussion}
We have used high-resolution cosmological simulations to illustrate that we
expect the ``local midplane'' defined by stellar density to vary with azimuth
by up to $\pm 100$ pc, as a natural consequence of the non-axisymmetry of the
Galactic disk at small scales. While this is not in itself surprising or new,
we also demonstrate that extending the coordinate system established by our
local midplane to a global axisymmetric coordinate system spanning the
entirety of the Galactic disk introduces systematic error in computations of
the actions under this symmetry assumption, when starting from the present-day
positions and velocities of stars as measured, for example, by \emph{Gaia}.
These systematic errors are most important for stars on thin-disk-like orbits,
where they can be large enough to yield actions representative of orbits in
the thick disk. This effect is entirely due to the extension of a local to a
global coordinate system, and is separate from real diffusion in stellar
integrals of motion caused by interactions with these same deviations from
axisymmetry, such as resonant perturbation by spiral arms or scattering from
molecular clouds.

This finding has many implications for the study of the dynamics of stars in
what is normally considered the regime of the epicyclic approximation:
quasi-harmonic oscillations around a circular ``guiding center orbit.'' Here
we consider a few of those implications, starting with an estimate of the
region around the Sun where we expect an extension of the local axisymmetric
coordinate system, and therefore the epicyclic approximation, to still be
valid. This region is also where it should be feasible to perform ``dynamical
tagging'' in the thin disk; that is, using the computed orbits or integrals of
motion of field stars to match them with known open clusters.

%%%move this later - decide where, probably conclusions
%While we focus our discussion now solely on the midplane, there is an
%additional complication in the measurement of the distance from the Sun to the
%Galactic center that may also be important. When modelling stars at Galactic
%radii close to the Sun's radius, it is in principle possible to determine what
%point they are orbiting about: the ``dynamical Galactic
%center.'' It is not guaranteed that this dynamical center will coincide
%with the distance to Sgr~A\textsuperscript{*} \citep{2018AA...615L..15G}. In
%fact, \citet{2015ApJ...803...80K} were able to measure the distance of the Sun
%to the Galactic center using the dynamical information of Palomar 5. Other
%dynamical measurements of the Solar distance have been made, though none with
%a precision capable of competing with the distance to Sgr~A\textsuperscript{*}
%\citep{1981gask.book.....M,2011PASJ...63..867S,2012MNRAS.427..274S,2013AstL...39...95B,2013IAUS..289..444Z}.

\subsection{The size of the Solar neighborhood} \label{sssec:neighborhood}
Given a local coordinate system, the amplitude and characteristic length scale
of the density fluctuations in the Galactic disk will determine how far this
system can be extended before introducing $\mathcal{O}(1)$ systematic errors
in the computation of actions for thin-disk stars. We consider this a
criterion for the region around the Sun where the analysis of the action-space
distribution of stars on quasi-circular orbits, and therefore the epicyclic
approximation, can be expected to be valid. This leads to a somewhat
dynamically motivated definition of the ``solar neighborhood.'' Since the
motivation may be different for defining the solar neighborhood in e.g.
studies of the chemical distribution of the Galaxy, we propose the phrase
``dynamical solar neighborhood.''

To illustrate this concept, we first compute the absolute range of midplane
values for a given $\Delta \phi$ segment of the midplane. With the observer's
azimuth set to zero, we consider a region extending from $-\Delta \phi/2$ to
$\Delta\phi/2$. To translate this angular scale into a physical scale, we also
compute the chord length $\lambda$ for the chord extending from the observer
to $\pm \Delta\phi/2$ (see Figure~\ref{fig:fig_to_explain}). We repeat this
for each of $50$ bins in azimuth. Each bin is plotted in
Figure~\ref{fig:range_deltaphi} as a gray line, with the mean value in solid
blue and the $1\,\sigma$ dispersion as the dashed blue lines.

\begin{figure*}
\plotone{fig/fig_to_explain.pdf}
\caption{A cartoon explanation of Figure~\ref{fig:range_deltaphi}. An observer
is placed at the blue x in the plane of a backgrround galaxy (lime). For a
given $\Delta \phi$ (blue) centered on the observer, we record the range of
midplane values for this section of the Solar circle (orange). We plot the
value of the range against the value of $\Delta \phi$. We are also able to
convert the value of $\Delta \phi$ into a chord length (pink), which we plot
as a secondary, upper $x$-axis. We repeat the procedure for each initial
$\phi$ (gray lines), and also compute the mean and $\pm1\,\sigma$ values (dark
blue).}
\label{fig:fig_to_explain}
\end{figure*}

\begin{figure*}
\plotone{fig/range_dphi.pdf}
\caption{The range of midplane heights encountered as a function of angular
width. At each angle $\phi$ from Figure~\ref{fig:midplane} we consider an
angular width of $\Delta \phi$ centered on $\phi$ and report the range of
midplane heights within that width. We repeat the procedure for each $\phi$
and plot the result as translucent gray lines. We also plot the mean range as
a solid blue line and the $\pm1\sigma$ lines as dashed blue lines. The upper
$x$-axis shows the chord length from the position $\phi$ to $\pm\Delta\phi/2$.
A cartoon explanation of this Figure is given in
Figure~\ref{fig:fig_to_explain}.}
\label{fig:range_deltaphi}
\end{figure*}

From Figure \ref{fig:range_deltaphi} it is clear that the range of midplane
values spanned in a given $\Delta \phi$ quickly saturates at $\Delta \phi
\gtrsim \pi/2$ radians (i.e. one quarter of the way around the Galactic disk
from the Sun). At far smaller $\Delta \phi$ it reaches the level of deviation
required for an $\mathcal{O}(1)$ systematic error in $J_z$ for our fiducial
thin-disk-like orbit (\thincolor horizontal line in Figure
\ref{fig:range_deltaphi}). For the three simulations considered, the median
range exceeds this value at $\Delta \phi >$ XX, YY, and ZZ radians, with
corresponding physical scales of $\lambda \sim$ XX, YY, and ZZ parsecs. These
can be considered estimates of the range of validity for an extension of the
locally determined axisymmetric coordinate system, at least when it comes to
the study of stars on quasi-circular orbits.

The situation is more forgiving for the study of thick-disk-like orbits, where
an $\mathcal{O}(1)$ systematic error in $J_z$ requires a larger midplane
deviation (\thickcolor horizontal line in Figure \ref{fig:range_deltaphi})
corresponding to $\Delta \phi >$ XX, YY, and ZZ radians or $\lambda \sim$ XX,
YY, and ZZ parsecs. For stars on halo-like orbits, extending the local
coordinate system globally is not likely to cause serious problems, since the
midplane deviations are relatively small compared to those needed to produce
$\mathcal{O}(1)$ systematic errors in the actions (\halocolor horizontal line
in Figure \ref{fig:range_deltaphi}). However, we caution that the tendency of
these midplane deviations is to \emph{inflate} the value of $J_z$ for thin
disk stars in an extremely non-Gaussian way (Figure \ref{fig:Jz_hist}),
scattering them into the region of action space normally associated with
different stellar populations and possibly contributing to confusion there.


\subsection{Consequences for disrupted cluster reconstruction}
\label{sssec:reconstruction}

We now consider the implications for attempts to reconstruct disrupting open
clusters by selecting ejected members based on their actions.

Suppose an open cluster of mass $m_c$ on an orbit with actions $\bm{J}_c$ is
disrupted. We would like to determine all stars in the {\em Gaia}
dataset\footnote{This is likely to be more successful with DR3/4 than DR2,
since upcoming data releases will contain a far greater number of stars with
radial velocities --- though targeted, follow-up radial velocity surveys may
allow this program to be performed sooner.} that are likely to be ejected
members from that cluster. One could generate an initial catalogue of
candidate members by integrating the cluster's orbit and selecting stars close
to that orbit, accounting for the fact that the stream will not exactly follow
the orbit of the cluster \citep[e.g.][]{2011MNRAS.413.1852E}.

One could then make a further selection by requiring that the action of each
field star $i$ be close to the cluster's actions to within some bound, i.e.
$\bm{J}_i - \bm{J}_c \in V_e$ where $V_e$ is some volume in action space
enclosing a region of acceptable error. A natural choice for $V_e$ is to
center it at the origin and allow its extent to be the intrinsic action-space
dispersion of a disrupted cluster. This dispersion can be estimated as
\citep[Section~8.3.3][]{2008gady.book.....B}
\beq \label{eq:action_disp}
\frac{\Delta J_i}{J_i} \sim \left(\frac{m_c}{M}\right)^{1/3}\text{,}
\eeq
where $m_c$ is the mass of the cluster and $M$ is the mass enclosed by the
cluster's orbit. A more sophisticated treatment of the action-space dispersion
of disrupted clusters can be obtained by following
\citet{2011MNRAS.413.1852E}, but such a treatment is premature for this work.

The act of comparing a field star's action to a cluster's action (i.e. in
computing $\bm{J}_i - \bm{J}_c$) is subject to significant error if the
cluster and field star have significantly different local midplanes and an
observer assumed a single global midplane.

To explore the magnitude of this effect, we consider the following program.
First, we assume a cluster mass $m_c$ and estimate the expected action-space
dispersion from Equation~\ref{eq:action_disp}. We then consult
Figure~\ref{fig:many_orbit_wrong_ref} to determine the $z$ offset necessary in
order for the induced error in the vertical action to be the same as the
expected action-space dispersion. If this were the case, an observer would
have to account for the different midplane of the field star and the cluster.
We perform this analysis for the thin-disk and thick-disk orbits considered in
Section~\ref{ssec:quant}, since the halo orbit is unrealistic for known open
clusters (Table~\ref{tab:real_clusters}).

With a characteristic $z$ offset in hand, we consult
Figure~\ref{fig:range_deltaphi} to convert this $z$ offset into a $\Delta
\phi$, which we convert to an arclength or characteristic distance, which we
refer to as the ``neighborhood radius'' or $R_{\n}$. Therefore, if an observer
is attempting to reconstruct a cluster with a given mass on a given orbit, we
will give a characteristic distance between a field star and the cluster
within which an observer might not need to consider differences in their
respective local midplanes.

\begin{figure}
\plotone{fig/cluster_offset.pdf}
\caption{The $z$ offset necessary in order for the action error induced by our
midplane effect to be comparable to the intrinsic action dispersion from a
cluster of mass $m_c$. We compute the intrinsic dispersion from
Equation~\ref{eq:action_disp} and convert the dispersion to a $z$ offset using
the result from Figure~\ref{fig:many_orbit_wrong_ref}. We consider the result
for the thin and thick orbit (Table~\ref{tab:orbits}). Nearby open clusters
(Table~\ref{tab:real_clusters}) have orbits closest to the thin orbit .}
\label{fig:cluster_offset}
\end{figure}

In Figure~\ref{fig:cluster_offset}, we show the $z$ offset necessary to induce
the action space dispersion computed from Equation~\ref{eq:action_disp}, using
the \texttt{gala} computed mass enclosed within a radius of $8.2\,\kpc$ ($\sim
10^{11}\,M_{\odot}$).

We now convert the $z$~offset-$m_c$ relation to an $R_{\n}$-$m_c$ relation,
where $R_{\n}$ indicates the neighborhood radius. An observer
attempting to reconstruct an open cluster of mass $m_c$ with perfect knowledge
of the midplane at the location of the cluster would expect to find the
ejected members using a dynamical cut for field stars within a distance
$R_{\n}$ of the cluster. If such an observer wants to find ejected stars that
are at a distance greater than $R_{\n}$, a more sophisticated analysis would
be required.

We use the relation in Figure~\ref{fig:range_deltaphi}, which shows the
typical $z$ offset at each $\Delta \phi$, to convert a given $z$~offset into a
typical $\Delta\phi$. We then convert this $\Delta \phi$ into a chord length
(see Figure~\ref{fig:fig_to_explain}) which we interpret as $R_{\n}$, the
neighborhood around the cluster. If the $z$~offset is greater than the maximum
range value of the galaxy, we simply report the maximum chord length
($2\times8.2\,\kpc$). The result of this procedure is shown in
Figure~\ref{fig:Rn_mc}. The solid lines indicate the neighborhood values
assuming the mean relation between $z$~offset and $m_c$, and the dashed lines
show the $+1\,\sigma$ and $-1\,\sigma$ relations, indicative of the strong
$\phi$-dependence of the midplane effect.

Figure~\ref{fig:Rn_mc} is most informative for the thin orbit (\thincolor).
The neighborhood around $\sim100\Msun$ clusters can be as small as
$\sim300\,\pc$ but up to $\sim1.2\,\kpc$ for \mi{} and \mf{}, though the
situation is marginally better for \mm{} with values ranging from
$\sim0.8\,\kpc$ to $\sim3\,\kpc$. The neighborhood values steadily increase
with $m_c$, topping out at $\sim5\,\kpc$ for $\sim10^4\,\Msun$ clusters in
\mi{} and \mf{}. In \mm{}, the neighborhood value reaches the maximum allowed of
$16.4\,\kpc$ for clusters of mass $\sim800\,\Msun$. The interpretation of this
is that the expected action space dispersion is so high for such high mass
clusters that the measured $z$~offset is not large enough throughout the
entire galaxy to be important.

For the thick-disk orbit (brown), the neighborhood is at the maximum value for
nearly all masses, except at the very low-mass end in \mi{}.

We repeat this entire procedure for each of the orbits of the real open
clusters in Table~\ref{tab:real_clusters} using the midplane variations from
\mi{}. We use \mi{} as a conservative choice, since it provides the greatest
midplane variation (Figure~\ref{fig:midplane}). We report the neighborhood
values for the mass of each cluster in the neighborhood column of
Table~\ref{tab:real_clusters}. Except for alphaPer, we find that the
neigbhorhood around each cluster is typically $\gtrsim 500\,\pc$. This
analysis assumes that the midplane from \mi{} can be accurately applied to
Milky Way orbits, which is not necessarily true. However, our analysis is
meant to be more illustrative than quantitative and these neighborhood numbers
should be understood as such.

\begin{figure*}
\plotone{fig/Rn_vs_mc.pdf}
\caption{The neighborhood around an open cluster of mass $m_c$ for the thin,
and thick orbits (\thincolor and \thickcolor, respectively; see
Table~\ref{tab:orbits}) and for each of the FIRE galaxies considered here ---
\mi{}, \mf{}, and \mm{} ({\em left}, {\em center}, and {\em right} panels,
respectively). Given the $z$ offset for a cluster of mass $m_c$ computed in
Figure~\ref{fig:cluster_offset}, we convert this to a chord length using the
result from Figure~\ref{fig:range_deltaphi}. We interpret this chord length as
the ``neighborhood'' around the cluster, i.e. the distance from the cluster
one could go before the $J_z$ error induced by our midplane effect is
comparable to the intrinsic action dispersion of the cluster. When the $z$
offset is larger than the maximum range in Figure~\ref{fig:range_deltaphi}, we
report the maximum chord length ($2\times8.2\,\kpc$). We aso report the same
procedure but using the $\pm1\sigma$ lines from
Figure~\ref{fig:range_deltaphi}, which we plot here as dashed lines. For the
thick orbit (\thickcolor), the neighborhood is quite large and thus the effect
is negligible. However, for the thin orbit (\thincolor), which is closest to
the orbits of nearby open clusters (see Table~\ref{tab:real_clusters}), the
neighborhood is only a few hundred $\pc$ for low-mass clusters.}
\label{fig:Rn_mc}
\end{figure*}

\section{Conclusions}\label{sec:conclusion}
Actions have promise as excellent orbit labels. If the Galaxy can be
approximated as axisymmetric and 6D phase space positions can be measured
accurately and precisely enough, then the computed actions are invariant with
orbital phase. However, we hvae shown that the fact that the Galactic midplane
is not constant across the disk presents a significant complication to
computed actions actually being invariant. Our main conclusions are as
follows:

\begin{itemize}
\item We demonstrated that an inaccuracy in the Galactocentric coordinate
system induces time evolution in the {\em observed} actions
(Figures~\ref{fig:cartoon}\&\ref{fig:one_orbit_wrong_ref}). 

\item We interpret this time evolution as an error in the computed action
(Figure~\ref{fig:many_orbit_wrong_ref}). We focused on the effect that an
inaccuracy in the midplane has on the vertical action $J_z$. We found that in
order for there to be a $100\%$ error in $J_z$ (as defined by the middle
$90\uth$ percentile) one needs a midplane error of $\sim30\,\pc$ and
$\sim225\,\pc$ for a typical thin and thick disk orbit, respectively. The
fractional error is significantly less for halo orbits.

\item We pointed out that the error distribution of the actions induced by a
coordinate system error is very non-Gaussian. The distribution is bimodal with
{\em neither of the modes peaking at the true value}. As a result, error
propagation is very involved when considering actions.

\item We proposed that dynamical modelling across large regions of the disk is
susceptible to this type of error, since the assumption that our local
Galactic midplane is the global Galactic midplane is not true {\em a priori}.

\item To demonstrate the previous point, we measured the local Galactic
midplane along the Solar circle in three different high-resolution, zoom-in
simulations of Milky Way mass galaxies from the FIRE collaboration. We found
that the midplane varies by $\sim185, 162, 84\,\pc$ (middle $90\%$) for the
three galaxies we considered (\mi{}, \mf{}, and \mm{}, respectively).

\end{itemize}

While in this work we have focused on errors in action computation, all of our
conclusions also extend to studies of stars that simply rely on orbit
integration, since the computation of actions and orbit integrations are
essentially equivalent.

\acknowledgments
We would like to thank Megan Bedell, Tobias Buck, David W. Hogg, and Kathryn
Johnston for many useful discussions. A.B. would like to thank Todd Phillips
for helpful discussions. A.B. was supported in part by the Roy \& Diana
Vagelos Program in the Molecular Life Sciences and the Roy \& Diana Vagelos
Challenge Award. The work of \Gus{everyone} is supported by the Simons
Foundation. \Gus{remove last sentence if everyone CCA affiliated}

\appendix
\section{Orbits} \label{app:orbits}
We plot the three orbits considered throughout the work
(Table~\ref{tab:orbits}) in Figure~\ref{fig:plot_orbits}.

\begin{figure*}
\plotone{fig/orbits.pdf}
\caption{The three orbits presented in Table~\ref{tab:orbits} and considered
throughout the work. We plot the thin, thick, and halo orbits in the {\em
left}, {\em center}, and {\em right} columns, respectively. The {\em upper}
row shows a plot of $x$ vs. $y$ while the {\em lower} row shows $R$ vs. $z$.}
\label{fig:plot_orbits}
\end{figure*}

\section{LSR Variations} \label{app:lsr}
We consider the variations of the LSR as a function of azimuth at the fiducial
solar circle ($R_{\odot} = 8.2\,\kpc$). At each azimuth, $\phi$, we take the
median velocity in cylindrical coordinates of all stars within $200\,\pc$ of
the position, following \citet{2018arXiv180610564S}. No best-fit subtraction
was performed as in Figure~\ref{fig:midplane}.

\begin{figure*}
\plotone{fig/lsr.pdf}
\caption{The local standard of rest (LSR) as a function of azimuth at the
fiducial Solar circle ($R_{\odot} = 8.2\,\kpc$). No best-fit subtraction is
performed here as we did in the case of the midplane
(Section~\ref{ssec:local_midplane}).}
\end{figure*}

\bibliography{references}

\end{document}
