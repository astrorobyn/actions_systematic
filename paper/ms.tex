\documentclass[twocolumn]{aastex62}

% \submitjournal{ApJ}

\shortauthors{Beane et al.}

\usepackage{graphicx}

\newcommand{\Gus}[1]{\textcolor{red}{#1}}

\newcommand{\Msun}{\text{M}_\odot}
\newcommand{\pc}{\text{pc}}
\newcommand{\kpc}{\text{kpc}}
\newcommand{\Myr}{\text{Myr}}
\newcommand{\Gyr}{\text{Gyr}}
\newcommand{\kms}{\text{km}/\text{s}}

\newcommand{\abs}[1]{\left| #1 \right|}
\newcommand{\z}{z}

\newcommand{\mi}{\texttt{m12i}}
\newcommand{\mf}{\texttt{m12f}}
\newcommand{\mm}{\texttt{m12m}}

\begin{document}

\title{Our Galactocentric Reference Frame Is Probably Wrong, and Not by a
Little}

\correspondingauthor{Angus Beane}
\email{abeane@sas.upenn.edu}

\author{Angus Beane}
\affil{Center for Computational Astrophysics, Flatiron Institute, 162 5th
Ave., New York, NY 10010, USA}
\affil{Department of Physics \& Astronomy, University of Pennsylvania, 209
South 33rd St., Philadelphia, PA 19104, USA}

\author{Many Helpful People}
\noaffiliation

\begin{abstract}

In an axisymmetric potential the 6D phase space of stars' positions and
velocities can be reduced to a 3D space of invariant actions. We show that if
one assumes the wrong galactocentric reference frame this can induce
artificial non-axisymmetries which manifest as time dependence in the computed
actions. However, the Milky Way is not axisymmetric in detail and so there are
no truly axisymmetric axes. We instead define the  ``optimally axisymmetric''
axes as the reference frame which minimizes the time dependence of the
computed actions. Standard methods for determining the galactocentric
reference frame typically involve finding the median height of stars in the
disk. Using cosmological zoom-in simulations from FIRE-2, we show that the
optimally axisymmetric reference frame can differ from standard methods of
calculating the galactocentric reference frame by nearly a scale height. This
work calls into question the use of actions as dynamical invariants unless
these reference frame issues can be resolved.

% However, it is not expected that actions are invariant in detail for the
% Milky Way. This in-invariance can come about from assuming the wrong
% Galactic potential, but it can also come about due to true
% non-axisymmetries. We show that actions can gain artificial time-dependence
% also by assuming the wrong reference frame. We argue that standard methods
% for determining the galactocentric reference frame are not gu

% it's spiral arms, bar(s), and gas clouds are all non-axisymmetric features
% that should induce long-term evolution in action space. However, these same
% features will cause a star's true orbit to deviate slightly from its orbit
% in an axisymmetric potential. If an observer assumes an axisymmetric
% potential when computing actions, this will induce an artificial short-term
% phase dependence of the actions, which we interpret as systematic error.
% Using the potential of cosmological zoom-in simulations of Milky Way mass
% galaxies from the FIRE collaboration (the Latte suite), we investigate the
% magnitude of this systematic error. We find XXX. Something something lower
% limit.

\end{abstract}

\keywords{actions}

\section{Introduction} \label{sec:intro}
Our understanding of the Milky Way is currently undergoing a revolution in the
era of {\em Gaia} DR2. Recent major discoveries include the remnants of a
major merger \citep{2018Natur.563...85H}, a phase-space spiral (\Gus{cites})
indicating local substructure infall (\Gus{cites}), a gap indicative of a dark
matter subtructure in GD1 \citep{2018arXiv181103631B}, and \Gus{others}.

Underlying all of these discoveries is the assumption of a galactocentric
reference frame \citep{2008gady.book.....B}. In angular
coordinates, the center of the galaxy is taken to be the location of
Sgr~A\textsuperscript{*} \citep[e.g.][]{2004ApJ...616..872R}. From stellar
motions near Sgr~A\textsuperscript{*}, the distance to the center is taken to
be $\sim 8.1\text{--}8.3\,\kpc$
\citep{2009ApJ...692.1075G,2018A&A...615L..15G}. The vertical height is found
by identifying where the stellar positions and velocities reach a maximum
(effectively the median height of all disk stars), and is usually taken to be
$\sim 25\,\pc$ \citep{2001ApJ...553..184C}, with more recent measurements from
{\em Gaia} DR2 placing it at $\sim 20\,\pc$ \citep{2019MNRAS.482.1417B}.

These measurements of the galactocentric reference frame rely on the
assumption of an axisymmetric Milky Way. Under this assumption, these
estimates would provide the correct parameters for the galactocentric
reference frame up to measurement error. However, the Milky Way is not
axisymmetric in detail. Spiral arms, bar(s), and infalling satellite galaxies
such as Sagittarius and the Large Magellanic Cloud all induce
non-axisymmetries.

Since the Milky Way is not truly axisymmetric, it is not clear the best way to
define the galactocentric reference frame. In this work, we consider one
possible physically motivated approach to defining the galactocentric
reference frame, and show that it can differ from conventional methods of
frame measurements.

In a slowly-evolving axisymmetric potential, the actions of stars' orbits are
conserved quantities \citep{2008gady.book.....B,2014RvMP...86....1S}. In other
words, if we could measure the actions of a star in such a potential, we would
measure the same actions over the course of the orbit. The non-axisymmetries
of a more realistic potential induce time evolution in the {\em measured}
actions --- the true orbital invariants (if they exist) are invariant by
definition. Through simple tests, we will see that even in a truly
axisymmetric potential, if the wrong reference frame is assumed then there
will be additional time evolution in the measured actions.

To provide one such realization of a realistic Galactic potential, we turn to
cosmological zoom-ins of Milky Way-mass galaxies from the Feedback in
Realistic Environments (FIRE)
collaboration\footnote{\url{https://fire.northwestern.edu}} (the Latte suite
of simulations).

\section{Reference Frame Gone Wrong} \label{sec:ref_frame}
We show and quantify the time evolution of measured actions that comes about
from inaccuracies in the galactocentric reference frame.

\subsection{Cartoon} \label{ssec:cartoon}
We begin with a cartoon illustrating what we mean in Figure~\ref{fig:cartoon}.
The solid gray line indicates a hypothetical ``true'' orbit. The y-axis
corresponds to the vertical height of the orbit and the x-axis to either time
or orbital phase. The dashed gray line is an incorrect midplane which an
observer uses to compute orbits.

Now suppose this observer makes a measurement of the true orbit at the orange
point or the red point. Then the orange and red lines correspond to the orbits
that the observer would compute for each point. In action space, this would
correspond to a different value of $J_z$ for the orange and red points. In
this way, assuming the wrong reference frame induces time dependence in the
actions {\em computed by an observer}.

\begin{figure*}
\plotone{fig/cartoon.pdf}
\caption{Caption.}
\label{fig:cartoon}
\end{figure*}

\subsection{Quantification} \label{ssec:quant}
We now turn to classical orbital integration in a commonly assumed Milky Way
potential to quantify the time variation of computed actions for different
reference frame errors. 



\section{Optimally Axisymmetric Frame} \label{sec:oa_frame}
An analytic, axisymmetric potential with a known reference frame is a far too
simple description of the Milky Way. We thus turn to cosmological zoom-in
simulations as approximations to a ``true'' Milky Way potential. We will
perform a few simulations of small clusters embedded in the potential of a
frozen $\z=0$ snapshot of one such zoom-in. We take our fiducial reference
frame from \citet{2018arXiv180610564S}.

Freezing the potential means that during the course of our simulation the
cluster will rapidly pass by much of the substructure in the potential. To
illustrate that this has little effect on our results, we resimulate each
cluster with the softening lengths of the embedded potential increased by a
factor of \Gus{something}.

Once we perform this simulation, we calculate the actions of each star in the
cluster assuming the best-fit smooth, axisymmetric potential, as we do
observationally for the Milky Way. We refer to this space as the observational
action space. We will find that there is significant time evolution in
observational action space. Attributing some of this evolution to our fiducial
reference frame being incorrect, we minimize the time-dispersion in action
space

\subsection{Cosmological Zoom-in} \label{ssec:cosmozoom}
Cosmological zoom-ins allow one to simulate a selected region at high
resolution embedded in a low-resolution cosmological background
\citep[e.g.][]{1993ApJ...412..455K,2014MNRAS.437.1894O}. Here we use a
zoom-in of a Milky Way-mass galaxy from the FIRE-2 collaboration.

First, a dark matter only simulation was run with $\Lambda$CDM cosmology
parameterized by $\Omega_m = 0.272$, $\Omega_{\Lambda} = 0.728$, $\Omega_b =
0.0455$, $h = 0.702$, $\sigma_8 = 0.807$, and $n_s = 0.961$, consistent with
current constraints \citep{2018arXiv180706209P}. Halos are selected at $z=0$
based solely on their mass ($\sim 1\text{--}2 \times 10^{12} \Msun$).
Particles within $5 R_{200\text{m}}$ of each halo are traced back to $\z=99$
and the initial conditions are regenerated at higher resolution using MUSIC
\citep{2011MNRAS.415.2101H}. A total of $\sim 10$ halos were simulated at
higher resolution \citep{2018MNRAS.481.4133G}.

These halos are uncontaminated and contain gas particles of mass $\sim 7000
\text{--} 20,000\,\Msun$ and star particles of mass $\sim 5000 \text{--} 7000\,
\Msun$, with the lower end coming from stellar evolution
\citep{2018arXiv180610564S}. Softening lengths for dark matter and star
particles are fixed at $112\,\pc$ and $11.2\,\pc$,
respectively.\footnote{This is $2.8$ times the often-quoted
Plummer-equivalent.} The gas softening length is adaptive, but at $z=0$ the
median softening length for gas particles around roughly solar positions
(with cylindrical radii within $500\,\pc$ of $8.2\,\kpc$ and
$\abs{z}<1\,\kpc$) is $98\,\pc$.

We restrict ourselves to the simulation \texttt{m12i} \citep[first introduced
in][]{2016ApJ...827L..23W}. For this work, this galaxy is the best-case
scenario. It does not contain a bar (like \texttt{m12m}) or companion galaxy,
and by eye is the closest to axisymmetric of the three galaxies presented in
\citet{2018arXiv180610564S} \citep{2018MNRAS.481.4133G}.

It is important to remember that simulations of Milky Way-mass galaxies are
not perfect representations of the true Milky Way, as discussed in
\citet{2018arXiv180610564S}. For instance, the velocity structure of
\texttt{m12i} is closer to M31 than the Milky Way. However, in this work we
are most interested in the global properties of the \texttt{m12i} potential,
and specifically in deviations from axisymmetry. From this perspective,
\texttt{m12i} is actually far more axisymmetric than we might expect of the
Milky Way. While \texttt{m12i} has prominent spiral arms, it lacks a bar and
a companion like the Large Magellanic Cloud. In future work, these additional
complications also need to be addressed.

\subsection{Cluster Simulations} \label{ssec:cluster_sim}
We simulate a small star cluster of $256$ stars in the frozen $\z=0$
potential of $\texttt{m12i}$. We use the code \texttt{ph4} within the
\texttt{AMUSE} framework \citep{2011ascl.soft07007P, 2013CoPhC.184..456P,
2013A&A...557A..84P} for the cluster integration. We use a softening length
of $0.01\,\pc$ which means we do not correctly solve the internal dynamics of
the cluster, but this is not important for our purposes. To couple the
gravitational forces of the frozen snapshot to the cluster stars, we use the
\texttt{BRIDGE} \citep{2007PASJ...59.1095F}, also implemented in
\texttt{AMUSE}. Our simulation runs for $2\,\Gyr$, allowing us to solve for
$\sim 10$ orbits.

To compute the forces from the frozen potential, we use the simple tree code
\texttt{pykdgrav}\footnote{\url{https://github.com/omgspace/pykdgrav},
commit:\texttt{fbd1ddc}}. 

We use the initial position and velocity of two star particles in
$\texttt{m12i}$

Even though we change the center, we do not recalculate the center velocity.
This is largely irrelevant, as the recalculated center velocity differs from
the fiducial center velocity by $<0.1\,\kms$. In recentering the galaxy, the
package \texttt{gizmo\_analysis} will subtract a slightly different value for
the Hubble flow from each particle's velocity to convert to a physical
reference frame. The error this introduces is $\sim 0.02\,\kms$, which we also
ignore.

\appendix \section{Appendix 1}
Gonna write an appendix probably 

\bibliography{references}

\end{document}
