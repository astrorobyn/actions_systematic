\documentclass[twocolumn]{aastex62}

% \submitjournal{ApJ}

\shortauthors{Beane et al.}

\newcommand{\Gus}[1]{\textcolor{red}{#1}}

\newcommand{\Msun}{\text{M}_\odot}
\newcommand{\pc}{\text{pc}}
\newcommand{\kpc}{\text{kpc}}

\newcommand{\mi}{\texttt{m12i}}
\newcommand{\mf}{\texttt{m12f}}
\newcommand{\mm}{\texttt{m12m}}

\begin{document}

\title{Actions are Bad, Mmkay}

\correspondingauthor{Angus Beane}
\email{abeane@sas.upenn.edu}

\author{Angus Beane}
\affil{Center for Computational Astrophysics, Flatiron Institute, 162 5th
Ave., New York, NY 10010, USA}
\affil{Department of Physics \& Astronomy, University of Pennsylvania, 209
South 33rd St., Philadelphia, PA 19104, USA}

\author{Many Helpful People}
\noaffiliation

\begin{abstract}

In an axisymmetric potential the 6D phase space of stars' positions and
velocities can be reduced to a 3D space of invariant actions. For this reason
actions are commonly used to help interpret phase space. However, it is not
expected that actions are invariant in detail for the Milky Way --- it's
spiral arms, bar(s), and gas clouds are all non-axisymmetric features that
should induce long-term evolution in action space. However, these same
features will cause a star's true orbit to deviate slightly from its orbit in
an axisymmetric potential. If an observer assumes an axisymmetric potential
when computing actions, this will induce an artificial short-term phase
dependence of the actions, which we interpret as systematic error. Using the
potential of cosmological zoom-in simulations of Milky Way mass galaxies from
the FIRE collaboration (the Latte suite), we investigate the magnitude of
this systematic error. We find XXX. Something something lower limit.

\end{abstract}

\keywords{actions}

\section{Introduction} \label{sec:intro}


We could have proceeded by taking an axisymmetric potential and adding
analytic spiral arms, a bar, and gas clouds. However, in preliminary work
with the FIRE galaxies we found that defining a consistent reference frame
between snapshots was very difficult. An inaccuracy in the direction of the
$z$-axis by a single degree induces a $140\,\pc$ change in the height of the
midplane at $8.2\,\kpc$. In a cosmological context it is not apparent that
there exists a consistent definition of the galactocentric frame, and we
therefore proceed in a cosmological context to include this effect.

We perform a set of three different simulations repeated for two of the Milky
Way zoom-ins, \mi and \mf.\footnote{The number corresponds to the mass of the
dark matter halo (here, $\sim 10^{12} \Msun$). The first ``m'' stands for
mass and the last letter is a unique identifier for the {\em initial
conditions}.} The first assumes an axisymmetric potential and also implicitly
a well-defined galactocentric frame. This first simulation shows that our
integration technique respects the fact that in an axisymmetric potential
actions are invariant and that our calculation of the actions does not
imprint some artificial phase dependence in the actions.


\section{Methods} \label{sec:methods}
We begin by 

Rockstar import in gizmo_analysis is commented out

\subsection{Simulated Galaxies} \label{ssec:fire}
We wish to perform N-body integration within a realistic Galactic potential.
In order to do this, we use the {\em Latte} suite of Milky Way-mass zoom-ins
from the Feedback in Realistic Environments (FIRE)
collaboration\footnote{\url{http://fire.northwestern.edu}}
\citep{2016ApJ...827L..23W,2018MNRAS.480..800H}. In a zoom-in cosmological
simulation a high resolution region of interest is embedded within a low
resolution cosmological context (\Gus{cites}). In the case of {\em Latte}, a
region that will evolve into a Milky Way-mass galaxy, identified in a
preliminary dark matter only run, is increased in resolution using MUSIC.
This allows one to achieve ultra-high resolution simulations of individual
galaxies than possible in a full cosmological run.

Our centering differs from Sanderson et al. by $\sim3.2\,\pc$ for \mi.


\section{Simulations}

\bibliography{references}

\end{document}
