\documentclass[twocolumn]{aastex62}

% \submitjournal{ApJ}

\shortauthors{Beane et al.}
\shorttitle{Our Galactic Midplane}

\usepackage{graphicx}
\usepackage{gensymb}
\usepackage{bm}

\newcommand{\Gus}[1]{\textcolor{red}{#1}}

\newcommand{\Msun}{\text{M}_\odot}
\newcommand{\pc}{\text{pc}}
\newcommand{\kpc}{\text{kpc}}
\newcommand{\Myr}{\text{Myr}}
\newcommand{\Gyr}{\text{Gyr}}
\newcommand{\kms}{\text{km}/\text{s}}
\newcommand{\actunit}{\text{kpc}\,\kms}

\newcommand{\abs}[1]{\left| #1 \right|}
\newcommand{\z}{z_r}
\newcommand{\uth}{\textsuperscript{th}}
\newcommand{\n}{\text{n}}

\newcommand{\beq}{\begin{equation}}
\newcommand{\eeq}{\end{equation}}

% affiliations
\newcommand{\cca}{Center for Computational Astrophysics, Flatiron Institute,
162 5th Ave., New York, NY 10010, USA}
\newcommand{\penn}{Department of Physics \& Astronomy, University of
Pennsylvania, 209 South 33rd St., Philadelphia, PA 19104, USA}
\newcommand{\amnh}{Department of Astrophysics, American Museum of Natural
History, Central Park West at 79th St., New York, NY 10024, USA}
\newcommand{\columbia}{Department of Astronomy, Columbia University, 550 W
120th St., New York, NY 10027, USA}

\begin{document}

\title{Our Galactic Midplane Is Probably Wrong, and Not by a Little}

% \correspondingauthor{Angus Beane}
\email{abeane@sas.upenn.edu}

\author{Angus Beane}
\affil{\cca}
\affil{\penn}

\author{Robyn Sanderson}
\affil{\cca}
\affil{\penn}

\author{Mordecai-Mark Mac Low}
\affil{\cca}
\affil{\amnh}

\author{Melissa K. Ness}
\affil{\cca}
\affil{\columbia}

\author{Daniel Angl\'es-Alc\'azar}
\affil{\cca}

\author{Megan Bedell}
\affil{\cca}

\begin{abstract}

In a completely axisymmetric potential, the notion of a global Galactic
midplane is well defined. However, since the Milky Way is not completely
axisymmetric, the global Galactic midplane is to some level of precision not
well defined. The local Galactic midplane has been found by determining the
height at which the number counts of stars reaches a maximum. This local
midplane is then used as the global midplane, an assumption which would be
justified in a completely axisymmetric potential. Assuming a global Galactic
midplane exists, we show in three cosmological zoom-in simulations from the
FIRE collaboration that local determinations of the midplane can vary by up to
a thin disk scale height. The effect is more pronounced in galaxies with less
quiescent histories (e.g. Sagittarius-like encounters). We explore one
consequence of such an inaccuracy of the midplane in the context of orbital
actions. Using standard orbit integration techniques, we show that midplane
inaccuracy induces artifical time evolution in the vertical action that makes
it virtually unusable for individual stars in the disk. The situation is less
dire for angular momentum and radial action determinations, though we also
show that inaccuracies in the solar radius can have a similar effect on the
latter. As a result, we question the use of actions as dynamical invariants in
the Milky Way disk.

\end{abstract}

\keywords{actions}

\section{Introduction} \label{sec:intro}
Our understanding of the Milky Way is currently undergoing a revolution in the
era of {\em Gaia} DR2. Recent major discoveries include the remnants of a
major merger \citep{2018Natur.563...85H}, a phase-space spiral (\Gus{cites})
indicating local substructure infall (\Gus{cites}), a gap indicative of a dark
matter subtructure in GD1 \citep{2018arXiv181103631B}, and \Gus{others}.

Underlying all of these discoveries is the assumption of a Galactocentric
coordinate system \citep{2008gady.book.....B}. In angular
coordinates, the center of the galaxy is taken to be the location of
Sgr~A\textsuperscript{*} \citep[e.g.][]{2004ApJ...616..872R}. From stellar
motions near Sgr~A\textsuperscript{*}, the distance to the center is taken to
be $\sim 8.1\text{--}8.3\,\kpc$
\citep{2009ApJ...692.1075G,2018A&A...615L..15G}. The vertical height is found
by identifying where the stellar positions and velocities reach a maximum
(effectively the median height of all disk stars), and is usually taken to be
$\sim 25\,\pc$ \citep{2001ApJ...553..184C}, with more recent measurements from
{\em Gaia} DR2 placing it at $\sim 20\,\pc$ \citep{2019MNRAS.482.1417B}.

These measurements of the galactocentric coordinate system rely on the
assumption of an axisymmetric Milky Way. Under this assumption, these
estimates would provide the correct parameters for the Galactocentric
coordinate system up to measurement error. However, the Milky Way is not
axisymmetric in detail. Spiral arms, bar(s), and infalling satellite galaxies
such as Sagittarius and the Large Magellanic Cloud all induce
non-axisymmetries.

We define the ``global Galactic midplane'' to be the plane of $z$ symmetry in
the axisymmetric model of the Milky Way. We define the ``local Galactic
midplane'' to be the plane of $z$ symmetry inferred by an observer at a
particular location. The true global Galactic midplane is the plane of $z$
symmetry in the best-fit axisymmetric model. It is not guaranteed that the
local Galactic midplane coincide with the global Galactic midplane. Even
without a high-precision definition and measurement of the global Galactic
midplane, we can still assess how different the two should be by measuring the
azimuthal dependence of the local Galactic midplane. This procedure may be
readily tractable with current datasets. However, we turn to simulation in
this work.

It's important for us to emphasize that when we state our local Galactic
midplane may be erroneous, we do not mean that past measurements/estimates of
it are wrong. Rather, what we mean is that it may vary from the local midplane
at other points in the Galaxy. In the face of {\em Gaia} DR2 (and soon DR3/4),
this may be an important effect to consider when performing dynamical
modelling across large distances.

Since an azimuthal-dependent midplane is effectively a departure from a
perfectly axisymmetric potential, we consider the effect that an erroneous
midplane has on the conservations of actions. The actions of an orbit are
typically computed in a cylindrical coordinate system as,
\beq\label{eq:actions}
J_i \equiv
\frac{1}{2\pi} \oint_{\text{orbit}}p_i\,\text{d}x_i\text{,}
\eeq
where $i=R,z,\phi$. Note that $J_{\phi} \equiv L_z$. In a slowly-evolving
axisymmetric potential, the actions of stars' orbits are conserved quantities
\citep{2008gady.book.....B,2014RvMP...86....1S}. The non-axisymmetries of a
more realistic potential induce time evolution in the {\em measured} actions
--- the true orbital invariants (if they exist) are invariant by definition.
Through simple tests, we will see that even in a truly axisymmetric potential,
if the wrong coordinate system is assumed then there will be additional time
evolution in the measured actions.

The transformation to action space is not trivial, and many of the assumptions
of the transformation are potentially incorrect. For instance, the assumption
that the Galactic potential is axisymmetric may be inadequate with current
measurement errors. Even if this assumption is valid, the parameters used in
our axisymmetric models may be inadequately tuned to observations. In
addition, the necessity of 6D phase space measurements of a star limits the
Galactic volume observable in action space.

With these limitations in mind, it is therefore natural to consider the reason
to transform to action space. One reason is dimensionality reduction --- in
the action space paradigm, a stellar orbit is completely described by three
actions, as opposed to six dimensions of phase space. This means that the
relationship between {\em orbital} properties of individual stars and stellar
properties such as age or metallicity can be investigated.

We explore one science goal that relies on the conservation of actions --- the
reconstruction of open clusters across large distances using dynamical
tagging. The basic program is to match stars with roughly the same actions as
their supposed natal cluster. To set the scene, we consider the orbits of the
nine open clusters within $250\,\pc$ as reported by the {\em Gaia}
collaboration \citep{2018A&A...616A..10G}. In Table~\ref{tab:real_clusters},
we report the distance to each cluster, the three actions, and the maximum
vertical extent of their orbits ($z_{\text{max}}$). The actions are computed
using the technique described in Section~\ref{ssec:action_comp}.

\begin{deluxetable*}{ccccccc}
\tablecaption{Relevant dynamical parameters for the $9$ open clusters within $250\,\pc$ of the Sun.}
\tablehead{\colhead{cluster} & \colhead{distance} & \colhead{$J_r$} &
\colhead{$L_z$} & \colhead{$J_z$} & \colhead{$z_{\text{max}}$} & \colhead{mass} \\ \colhead{ } &
\colhead{$\mathrm{pc}$} & \colhead{$\mathrm{kpc\,km\,s^{-1}}$} &
\colhead{$\mathrm{kpc\,km\,s^{-1}}$} & \colhead{$\mathrm{kpc\,km\,s^{-1}}$} &
\colhead{$\mathrm{pc}$} & \colhead{$M_{\odot}$}}f
\startdata
alphaPer & 174.9 & 14.0 & -1739 & 0.0045 & 10.5 & 352\tablenotemark{a} \\
Blanco1 & 237.2 & 1.6 & -1841 & 1.53 & 207.0 & 410\tablenotemark{b} \\
ComaBer & 85.9 & 1.6 & -1860 & 0.71 & 139.7 &  \\
Hyades & 47.5 & 20.3 & -1757 & 0.26 & 81.5 & 400\tablenotemark{d} \\
IC2391 & 151.6 & 5.5 & -1790 & 0.018 & 21.2 & \\
IC2602 & 152.2 & 11.5 & -1718 & 0.32 & 88.6 & \\
NGC2451 & 193.7 & 5.8 & -1805 & 0.21 & 73.4 & \\
Pleiades & 135.8 & 19.3 & -1696 & 0.38 & 96.3 & \\
Praesepe & 186.2 & 21.2 & -1766 & 0.58 & 122.9 & 
\enddata

\tablenotetext{a}{\citet{2016MNRAS.457.1028S}}
\tablenotetext{b}{\citet{2007AA...471..499M}}
\tablenotetext{c}{}
\tablenotetext{d}{\citet{1998AA...331...81P}}


\tablerefs{\citet{2018AA...616A..10G}}
\label{tab:real_clusters}
\end{deluxetable*}

To provide a realization of a realistic Galactic potential, we turn to
cosmological zoom-ins of Milky Way-mass galaxies from the Feedback in
Realistic Environments (FIRE)
collaboration\footnote{\url{https://fire.northwestern.edu}} (the {\em Latte} suite
of simulations).

In Section~\ref{sec:ref_frame}, we describe the general impact coordinate
system errors have on the measured actions. In Section~\ref{sec:local_fire},
we measure the local midplane in cosmological zoom-in simulations from the
FIRE collaboration. In Section~\ref{sec:discussion} we discuss and give some
context for our results, and implications for the definition of the Solar
neighborhood. We conclude in Section~\ref{sec:conclusion}.

\section{Impact on Actions} \label{sec:ref_frame}
We first demonstrate the importance of measuring the Galactic midplane and
Solar radius to very high accuracy. To be clear, the consequences we explore
here may also arise from various other observational errors. For instance, the
symmetrized Galactic potential used by observers may not be a good description
of the true potential --- or the parameters used may be incorrect. In this
Section, we assume these approximations are valid and simply explore the
consequences of an inaccurate Galactocentric coordinate system.

\subsection{Cartoon} \label{ssec:cartoon}
We present a cartoon in Figure~\ref{fig:cartoon} to show that an inaccurate
determination of the midplane can lead to a phase-dependence of the {\em
observed} actions. The solid gray line indicates a hypothetical ``true''
orbit. The y-axis corresponds to the vertical height of the orbit and the
x-axis orbital phase. The dashed gray line is an incorrect midplane which an
observer uses to compute orbits.

Now suppose this observer makes a measurement of the true orbit at the blue
point or the pink point. Then the blue and pink lines correspond to the orbits
that the observer would compute for each point. In action space, this would
correspond to a different value of $J_z$ for the blue and pink points. In this
way, assuming the wrong coordinate system induces time dependence in the
actions {\em computed by an observer}.

\begin{figure*}
\plotone{fig/cartoon.pdf}
\caption{Cartoon showing the effect an incorrect midplane can have on orbit
integrations and as an extension on action computations. The x-axis shows the
orbital phase and the y-axis the vertical height. The top gray curve depicts a
hypothetical true orbit oscillating about the true midplane, the horizontal
solid gray line. Consider an immortal observer who erroneously assumes the
midplane is located at the horizontal dashed line. Suppose this immortal
observer measures the orbit at the blue and pink points. If the immortal
observer were to perform orbit integrations at each observation, they would
assume the blue and pink curves for the respective orbits. In this way, an
incorrect midplane will induce phase-dependence in the {\em observed}
actions.}
\label{fig:cartoon}
\end{figure*}

\subsection{Summary of Action Computation} \label{ssec:action_comp}
We follow our previous work in computing actions \citep{2018ApJ...867...31B}.
We use the code \texttt{gala} v0.3 to perform orbit integrations and
conversion to action space \citep{2017JOSS....2..388P,Price-Whelan:2018},
which uses an action estimator technique presented by
\citet{2014MNRAS.441.3284S}. We use the default \texttt{MWPotential} as our
potential, which is based on the Milky Way potential available in
\texttt{galpy} \citep{2015ApJS..216...29B}. This potential includes a
Hernquist bulge and nucleus \citep{1990ApJ...356..359H}, a Miyamoto-Nagai disk
\citep{1975PASJ...27..533M}, and an NFW halo \citep{1997ApJ...490..493N}, and
is fit to empirically match some observations. We use the Dormand-Prince
8(5,3) integration scheme \citep{Dormand80:integrator}. We use a timestep of
$1\,\Myr$ and integrate for $5\,\Gyr$, corresponding to $\sim 20$ orbits for a
Sun-like star.

Other methods for computing actions are used in the literature. For example,
the St\"ackel Fudge method \citep{2016MNRAS.457.2107S} which uses the
St\"ackel potential to approximate the Galactic potential
\citep{1985MNRAS.216..273D,2012MNRAS.426.1324B} is commonly used
\citep{2018arXiv180503653T,2018MNRAS.481.4093S,2018arXiv180803278T}.

\subsection{Quantification} \label{ssec:quant}
First, we illustrate the time-dependence of one possible inaccurate reference
frame. Throughout this work we will consider three orbits typical of the
stellar thin disk, stellar thick disk, and the stellar halo. We summarize
their initial position in phase space and the actions of their integrated
orbits in Table~\ref{tab:orbits}. Each orbit is plotted in \Gus{Appendix X}.
We will refer to these orbits by their names henceforth.

We begin by considering the thick orbit. Suppose that an observer makes
observations of this orbit in a coordinate system that is incorrect in height
by $100\,\pc$ --- i.e. we subtract the vector $(0, 0, 100)\,\pc$ from each
position in the orbit. This corresponds to an observer physically located at
the position $(8, 0, 0)\,\kpc$ but erroneously thinking they are located at
$(8, 0, 0.1)\,\kpc$.

\begin{deluxetable*}{ccccccc}
\tablecaption{Description and names of the three orbits considered in this work.}
\tablehead{\colhead{name} & \colhead{initial position} & \colhead{initial
velocity} & \colhead{$J_r$} &
\colhead{$L_z$} & \colhead{$J_z$} & \colhead{$z_{\text{max}}$}\\ \colhead{ } &
\colhead{$\mathrm{kpc}$} & \colhead{$\mathrm{km}/\mathrm{s}$} &
\colhead{$\mathrm{kpc\,km\,s^{-1}}$} &
\colhead{$\mathrm{kpc\,km\,s^{-1}}$} & \colhead{$\mathrm{kpc\,km\,s^{-1}}$} &
\colhead{$\mathrm{pc}$}}
\startdata 
thin & $(8, 0, 0)$ & $(0, -190, 10)$ & 40.3 & -1520 & 0.69 & 122 \\
thick & $(8, 0, 0)$ & $(0, -190, 30)$ & 32.5 & -1520 & 23.0 & 425 \\ 
halo & $(8, 0, 0)$ & $(0, -190, 50)$ & 32.8 & -1520 & 529.1 & 854
\enddata
\label{tab:orbits}
\end{deluxetable*}

Using the observations of the orbit that this erroneous (and immortal)
observer makes every $\Myr$, we perform the same orbit integration procedure
as before to compute the actions our observer would observe in the incorrect
coordinate system over the duration of the orbit. The erroneous computed
actions are shown for the first $\Gyr$ of the orbit in the {\em upper} panels
Figure~\ref{fig:one_orbit_wrong_ref}. Occasionally the numerical scheme fails
and very large actions are reported by \texttt{gala}
--- we perform a $5\sigma$ clip on each action to exclude such orbits, but
    this only excludes a total of $5$ orbits out of the $1000$ considered for
    Figure~\ref{fig:one_orbit_wrong_ref}. Some numerical artifacts remain, but
    the vast majority of orbits are computed properly. We perform the same
    procedure but assume a $100\,\pc$ offset in the $x$ component (i.e.
    subtracting the vector $(100, 0, 0)\,\pc$) in the {\em lower} panels. This
    is equivalent to an observer making an incorrect measurement of the Solar
    radius.

Figure~\ref{fig:one_orbit_wrong_ref} shows that the erroneous observer would
infer significant time evolution in their computed actions. For an error in
$z$ ({\em upper} panels), the 95\textsuperscript{th} minus
5\textsuperscript{th} percentiles are $2.2$ and $6.2 \actunit$ for $J_r$ and
$J_z$, respectively. These are fractional errors of $5.7\%$ and $85.7\%$,
respectively. The error induced in $L_z$ is negligible, as expected since
$L_z$ does not depend on the value of $z$. It is worth pointing out that a
$100\,\pc$ error in an orbit with $z_{\text{max}}=425\,\pc$ --- a $24\%$ error
--- induced an $85.7\%$ error in the computation of $J_z$.

\Gus{For an error in the $x$ component (or Solar radius)}, the
95\textsuperscript{th} minus 5\textsuperscript{th} percentiles are $x$, $y$,
and $z \actunit$ for $J_r$, $L_z$ and $J_z$, respectively. These are
fractional errors of $x\%$, $y\%$ and $z\%$, respectively.

\begin{figure*}
\plotone{fig/schmactions_one_orbit.pdf}
\caption{The artificial phase-dependence in the observed actions induced by an
error in the Galactocentric coordinate system. We consider here the thick disk
orbit, which has actions of $(J_r, L_z, J_z) = (37.9, -1520, 7.0)\,\actunit$
and $z_{\text{max}}=425\,\pc$ (see Table~\ref{tab:orbits}). We integrate the
orbit according to the procedure laid out in Section~\ref{ssec:action_comp}.
Then, we subtract $100\,\pc$ from the $z$ value ({\em upper panels}) or the
$x$ value ({\em lower panels}) of each position in the orbit, corresponding to
an erroneous observer assuming a midplane ({\em upper}) or solar radius ({\em
lower}) that is off by $100\,\pc$. We then allow our (immortal) observer to
observe the orbit over $1\,\Gyr$ and perform the same orbit integration
procedure at each timestep, and report the values of the actions. The
computation of $L_z$ is pristine to errors in $z$, with only numerical
artifacts remaining. Only small errors are induced in $J_r$, with the middle
$90\%$ of values over the $\Gyr$ being $\sim6\%$ of the true $J_r$. As
expected, large errors are induced in $J_z$, with the middle $90\%$ of values
being $\sim85\%$ of the true $J_z$.}
\label{fig:one_orbit_wrong_ref}
\end{figure*}

We now repeat the procedure but continuously changing the error in the $z$ and
$x$ components. In Figure~\ref{fig:many_orbit_wrong_ref}, we report the 95\uth
minus 5\uth percentile divided by the true action value for three different
orbits. The \Gus{orange} orbit is exactly the same as in
Figure~\ref{fig:one_orbit_wrong_ref}, and we also consider two more planar
orbits with initial vertical velocities of $50\,\kms$ (\Gus{red}) and
$10\,\kms$ (\Gus{blue}). For these two new orbits, the true actions are $(J_r,
L_z, J_z) = (32.6, -1520, 22.9)\,\actunit$ and $(40.4, -1520, 0.7)\,\actunit$
for the \Gus{red} and \Gus{blue} orbits, respectively. The value of
$z_{\text{max}}$ for these orbits are $854\,\pc$ and $122\,\pc$, respectively
(with the orange orbit having $z_{\text{max}}$ of $425\,\pc$.) \Gus{Easy way
to explain change in $J_r$?}

The top row of Figure~\ref{fig:many_orbit_wrong_ref} shows the error in each
action assuming an offset in the $z$ component (i.e. the Galactic midplane).
In the {\em bottom row} we also consider offsets in the $x$ component (i.e.
the Solar radius). The {\em left}, {\em middle}, and {\em right} columns show
the relative error in $J_z$, $L_z$, and $J_r$, respectively. We compute the
relative error as the  95\uth minus 5\uth percentile of the measured action
($\Delta J_i$) over the course of the first $\Gyr$ of the incorrectly observed
orbit divided by the true action value as a percentage.

In the {\em upper middle} panel, there is essentially no error in the
determination of $L_z$. This is expected since, in principle, $L_z$ only
depends on the $x$- and $y$-components of the position and velocity of the
stars\footnote{Though this is not how we compute $L_z$ in practice.}, and is
thus unaffected by offsets in $z$. This also agrees with the result we found
in Figure~\ref{fig:one_orbit_wrong_ref}.

The {\em upper right} panel shows that the fractional error in $J_z$ is more
exaggerated for more planar orbits. For the nearly planar (blue) orbit,
deviations that are $\sim30\,\pc$ already give $100\%$ deviations in the
actions. The required offset for $100\%$ deviation is $\sim120$ and
$\sim225\,\pc$ for the orange and red orbits, respectively. \Gus{Make these
numbers exact.}

For the offset in the Solar radius {\em bottom panels}, the error is largest
for $J_r$, with some deviations resulting in $L_z$ and relatively small
deviations in $J_z$. In the {\em bottom left} panel the orange and blue lines
overlap, and in the {\em bottom middle} and {\em bottom right} panels all
three lines overlap.

\begin{figure*}
\plotone{fig/schmactions_many_orbits.pdf}
\caption{\Gus{update figure} We report the 95\uth minus 5\uth percentile of
the error in the measured action ($\Delta J_i$) from coordinate system errors
for three orbits with low-, medium-, and high-$J_z$ values (blue, orange, and
red, respectively). The {\em left}, {\em center}, and {\em right} panels show
the result for $J_r$, $L_z$, and $J_z$, respectively. The {\em upper} panels
consider an offset in $z$ and the {\em lower} panels consider an offset in $x$
(equivalently, an offset in the Solar radius).}
\label{fig:many_orbit_wrong_ref}
\end{figure*}

\begin{figure}
\plotone{fig/schmactions_Jz_hist.pdf}
\caption{Caption.}
\label{fig:Jz_hist}
\end{figure}

\section{Local Midplane in FIRE} \label{sec:local_fire}
\subsection{Description of Simulation} \label{ssec:cosmozoom}
Here we briefly describe the cosmological zoom-ins used in this work.
Cosmological zoom-ins allow one to simulate a selected region at high
resolution embedded in a low-resolution cosmological background
\citep[e.g.][]{1993ApJ...412..455K,2014MNRAS.437.1894O}. The FIRE
collaboration has simulated a number of Milky Way-mass zoom-ins as part of the
{\em Latte} suite of FIRE-2 simulations
\citep{2016ApJ...827L..23W,2018MNRAS.481.4133G}. We use the three zoom-ins
considered in \citet{2018arXiv180610564S} --- m12i, m12f, m12m, since they
have associated mock {\em Gaia} DR2 catalogues and the $z=0$ snapshots are
publicly available.

First, a dark matter only simulation was run with $\Lambda$CDM cosmology
parameterized by $\Omega_m = 0.272$, $\Omega_{\Lambda} = 0.728$, $\Omega_b =
0.0455$, $h = 0.702$, $\sigma_8 = 0.807$, and $n_s = 0.961$, consistent with
current constraints \citep{2018arXiv180706209P}. Halos are selected at $\z=0$
based solely on their mass ($\sim 1\text{--}2 \times 10^{12} \Msun$). In this
work, to avoid confusion with the vertical height $z$, we refer to
cosmological redshift as $\z$. Particles within $5 R_{200\text{m}}$ of each
halo are traced back to $\z=99$ and the initial conditions are regenerated at
higher resolution using MUSIC \citep{2011MNRAS.415.2101H}.

Simulations were run using
\texttt{GIZMO}\footnote{\url{http://www.tapir.caltech.edu/~phopkins/Site/GIZMO.html}}
\citep{2015MNRAS.450...53H}. Hydrodynamics is solved using the meshless
finite-mass method and gravity is solved using a modified version of the
Tree-PM solver of \texttt{GADGET-3}, using fully adaptive and fully
conservative gravitational force softening for gas.

These halos are uncontaminated and contain gas particles of mass $\sim 7000
\text{--} 20,000\,\Msun$ and star particles of mass $\sim 5000 \text{--} 7000\,
\Msun$\footnote{In fact, the simulation m12i contained a gas
splitting bug which causes a small number of gas and star particles to have
larger than typical masses. These constitute only $\sim0.2\%$ of the star
particles in the $z=0$ snapshots, and are only a factor of a few more massive
than typical star particles. Since our results are broadly consistent between
the three simulations, we ignore this minor complication.}, with the lower end
coming from stellar evolution \citep{2018arXiv180610564S}. Softening lengths
for dark matter and star particles are fixed at $112\,\pc$ and $11.2\,\pc$,
respectively.\footnote{This is $2.8$ times the often-quoted
Plummer-equivalent.} The gas softening length is adaptive, but at $z=0$ the
median softening length for gas particles around roughly solar positions (with
cylindrical radii within $500\,\pc$ of $8.2\,\kpc$ and $\abs{z}<1\,\kpc$) is
$98\,\pc$.

In this work, we take the coordinate systems used in
\citet{2018arXiv180610564S} as our fiducial coordinate systems for each
galaxy. The center of the galaxy is found through an iterative ``shrinking
spheres'' method, where the center is calculated for a given radius which is
then reduced by $50\%$ until converging on the galaxy's center. The center of
mass velocity is then determined by all star particles within $15\,\kpc$ of
this center. The galaxy is then rotated onto the principal axis frame as
determined by stars younger than $1\,\Gyr$ inside of the fiducial solar radius
$R_{\odot} = 8.2\,\kpc$.

It is important to remember that simulations of Milky Way-mass galaxies are
not perfect representations of the true Milky Way, as discussed in
\citet{2018arXiv180610564S}. For instance, the velocity structure of m12i is
closer to M31 than the Milky Way. However, in this work we are most interested
in the global properties of the m12i potential, and specifically in deviations
from axisymmetry. From this perspective, m12i is actually far more
axisymmetric than we might expect of the Milky Way. While m12i has prominent
spiral arms, it lacks a bar and a companion like the Large Magellanic Cloud.
In future work, these additional complications also need to be addressed.

\subsection{Local Midplane} \label{sec:local_midplane}
We now turn to defining the local midplane that an observer might measure if
they were situated in each of these galaxies. We place our imaginary observer
at the solar radius of $8.2\,\kpc$ and azimuth $\phi$\footnote{Our erroneous,
immortal observer is also warp-drive capable.}, and consider stars within a
cylinder of radius $0.5\,\kpc$ and height $1\,\kpc$ perpendicular to the
fiducial disk.

This median height is taken to be what our observer would measure as the local
Galactic midplane at each $\phi$, and is reported for each galaxy in
Figure~\ref{fig:midplane}. We find that only $10$ iterations of this procedure
are necessary for convergence. We bootstrap resample $1000$ times and report
the $95^{\text{th}}$ minus $5^{\text{th}}$ percentile as the dashed-line error
bars. This bootstrap resampling is performed by first selecting all stars
within a height of $2\,\kpc$, resampling that selection, and then repeating
the $10$ iterations.

To allow for potential small inaccuracies in the fiducial coordinate systems,
we also subtract the best fit $\text{A} \sin{\left(\phi + \text{B}\right)} +
\text{C}$ curve from the midplane as a function of azimuth. For $\text{A}$,
the values are $-165$, $45$, $9\,\pc$ and for $\text{C}$ the values are $-69$,
$19$, and $-18\,\pc$ for m12i, m12f, and m12m, respectively \Gus{Double check
numbers}. For the assumed solar radius of $8.2\,\kpc$, we can approximate the
angle offset for the $z$-axis from the values of $\text{A}$ --- we compute
\Gus{$1.15$, $0.31$, and $0.06\,\deg$} for m12i, m12f, and m12m. These angle
offsets are consistent with the values given in \citet{2018arXiv180610564S}
for the difference between the $z$-axis as defined by the gas and stars.

Figure~\ref{fig:midplane} shows the inferred midplane our imaginary observer
would make as a function of azimuth for each galaxy. The $90\%$ interquartile
range for each galaxy is $185$, $162$, $84\,\pc$ for m12i, m12f, and m12m. 

\begin{figure*}
\plotone{fig/midplane_fit.pdf}
\caption{The local midplane determined at the fiducial Solar radius
($8.2\,\kpc$) for the three FIRE galaxies m12i, m21f, and m12m ({\em left},
{\em center}, and {\em right} panels). The local midplane is determined at a
position $\phi$ by taking the median height of all stars within $R=0.5\,\kpc$
and $z=1\,\kpc$ (in cylindrical coordinates). The procedure is performed again
using the new height $10$ times to converge on the local midplane height. In
order to allow for the possibility that the fiducial Galactocentric coordinate
system is incorrect, we subtract the best fit $A\sin{(\phi+B)}+C$ curve from
each panel --- this figure is reproduced with the original midplane
determination (i.e. before subtracting the best fit sine curve) in
Appendix~\Gus{x}. We then bootstrap resample $1000$ times on all stars within
a $2\,\kpc$ height of the fiducial midplane to determine error bars (95\uth
and 5\uth percentiles), which we report as dashed lines.}
\label{fig:midplane}
\end{figure*}

\section{Discussion} \label{sec:discussion}
We have shown that, in simulation, the ``local midplane'' can have strong
azimuthal dependence. The implications of this finding are not clear, but
clearly demonstrate that assuming our local midplane is the same as the global
midplane is insufficient for extracting the most dynamical information from
the {\em Gaia} mission. From now on, we set aside the discussion of
inaccuracies in the Solar radius. While all that we state about the midplane
applies to the Solar radius as well, it is not clear the best way to determine
the dynamical center of the galaxy in the FIRE simulations.

\subsection{Applicability to the Milky Way} \label{ssec:is_it_real}
As we briefly discussed in Section~\ref{ssec:cosmozoom}, cosmological zoom-in
simulations of Milky Way-mass galaxies are not perfect representations of the
Milky Way, and do not reproduce every available observation of the Milky Way.
For instance, the velocity dispersion-age relation in m12i, m12f, and m12m is
closer to that of M31 \citep{2018arXiv180610564S}.

In order to assess the applicability of our work to the real Milky Way, we
consider a subset of Table~3 from \citet{2018arXiv180610564S} --- our
Table~\ref{tab:scale_height} shows the stellar scale heights of m12i, m12f,
and m12m and the current best estimates for the Milky Way. One noticeable
feature is that the FIRE galaxies are more ``puffy'' than the Milky Way.
\Gus{Maybe FIRE collaboration has some insight into the mechanism for this?}.
This would seem to suggest that the midplane estimates from
Figure~\ref{fig:midplane} over-estimate the true variation in the Milky Way.

\begin{deluxetable}{ccc}
\tablecaption{Stellar Disk Scale Heights}
\tablehead{\colhead{galaxy} & \colhead{thin disk} & \colhead{thick disk} \\ 
\colhead{} & \colhead{(pc)} & \colhead{(pc)} } 
\startdata
MW\tablenotemark{a} & 300 & 900 \\
m12i\tablenotemark{b} & 480 & 2000 \\
m12f\tablenotemark{b} & 440 & 1280 \\
m12m\tablenotemark{b} & 290 & 1030 \\
\enddata
% \tablerefs{\citet{2016ARAA..54..529B,2018arXiv180610564S,2008ApJ...673..864J}}

\tablenotetext{a}{\citet{2008ApJ...673..864J,2016ARAA..54..529B}}
\tablenotetext{b}{\citet{2018arXiv180610564S}}

\label{tab:scale_height}
\end{deluxetable}

Another important aspect to consider when comparing the simulated galaxies to
the Milky Way is the environment/merger history. For instance, m12i and m12m
both lack a massive neighbor analogous to the LMC. On the other hand, m12f has
had a recenter merger with \Gus{something}. \Gus{Expand more here - how to
incorporate merger trees?}

\subsection{Solar Neighborhood} \label{ssec:neighborhood}
To gain further handle on the midplane behavior as a function of azimuth, we
compute the range of midplane values for a given $\Delta \phi$ segment of the
midplane in Figure~\ref{fig:range_deltaphi}. With the observer's azimuth set
to zero, we consider a region extending from $-\Delta \phi/2$ to
$\Delta\phi/2$. We also compute the chord length for the chord extending from
the observer to $\pm \Delta\phi/2$. We repeat this for each of the $50$ bins
in azimuth. Each bin is plotted as a gray line, with the mean value in solid
blue and the $1\,\sigma$ dispersion as the dashed blue lines.

\begin{figure*}
\plotone{fig/range_dphi.pdf}
\caption{The range of midplane heights encountered as a function of angular
width. At each angle $\phi$ from Figure~\ref{fig:midplane} we consider an
angular width of $\Delta \phi$ centered on $\phi$ and report the range of
midplane heights within that width. We repeat the procedure for each $\phi$
and plot the result as translucent gray lines. We also plot the mean range as
a solid blue line and the $\pm1\sigma$ lines as dashed blue lines. The upper
$x$-axis shows the chord length from the position $\phi$ to
$\pm\Delta\phi/2$.}
\label{fig:range_deltaphi}
\end{figure*}

A possible interpretation of our finding is that one should use a star's local
midplane in order to compute dynamical quantities like the vertical action.
Under the assumption that this interpretation is correct, we now consider the
implications for dynamical studies of disrupting open clusters (the original
motivation of this study).

Suppose an open cluster of mass $m_c$ on an orbit with actions $\bm{J}_c$ is
disrupted, and that the cluster is currently at an orbit phase of
$\bm{\theta}_0$. We would like to determine all stars in the {\em Gaia}
dataset\footnote{This is likely to be more successful with DR3/4 than DR2,
since upcoming data releases will contain a far greater number of stars with
radial velocities --- though targeted, follow-up radial velocity surveys may
allow this program to be performed sooner.} that are likely to be ejected
members from that cluster. One could proceed by integrating the cluster's
orbit and selecting all stars within a certain distance of that orbit,
accounting for the fact that the stream will not exactly follow the orbit of
the cluster \citep[e.g.][]{2011MNRAS.413.1852E}.

One could then make a further selection by requiring that the action of each
field star $i$ be close to the cluster's actions to within some bound, i.e.
$\bm{J}_i - \bm{J}_c \in V_e$ where $V_e$ is some volume in action space
enclosing a region of acceptable error. A natural choice for $V_e$ is to
center it at the origin and allow its extent to be the intrinsic action-space
dispersion of a disrupted cluster. This dispersion can be estimated as
\citep[\S~8.3.3][]{2008gady.book.....B}
\beq \label{eq:action_disp}
\frac{\Delta J_i}{J_i} \sim \left(\frac{m_c}{M}\right)^{1/3}\text{,}
\eeq
where $m_c$ is the mass of the cluster and $M$ is the mass enclosed by the
cluster's orbit. A more sophisticated treatment of the action-space dispersion
of disrupted clusters can be obtained by following
\citet{2011MNRAS.413.1852E}, but such a treatment is premature for this work.

The act of comparing a field star's action to a cluster's action (i.e. in
computing $\bm{J}_i - \bm{J}_c$) is subject to significant error if the
cluster and field star have significantly different local midplanes and an
observer assumed a single global midplane.

To explore the magnitude of this effect, we consider the following program.
First, we assume a cluster mass $m_c$ and estimate the expected action-space
dispersion from Equation~\ref{eq:action_disp}. We then consult
Figure~\ref{fig:many_orbit_wrong_ref} to determine the $z$ offset necessary in
order for the induced error in the vertical action to be the same as the
expected action-space dispersion. If this were the case, an observer would
have to account for the different midplane of the field star and the cluster.
We repeat this analysis for each of the three orbits considered in
Section~\ref{ssec:quant}.

With a characteristic $z$ offset in hand, we consult
Figure~\ref{fig:range_deltaphi} to convert this $z$ offset into a $\Delta
\phi$, which we convert to an arclength or characteristic distance (the
difference between the two will be negligible for the small $\Delta \phi$'s we
will compute). Therefore, if an observer is attempting to reconstruct a
cluster with a given mass on a given orbit, we will give a characteristic
distance between a field star and the cluster within which an observer might
not need to consider differences in their respective local midplanes.

\begin{figure}
\plotone{fig/cluster_offset.pdf}
\caption{The $z$ offset necessary in order for the action error induced by our
midplane effect to be comparable to the intrinsic action dispersion from a
cluster of mass $m_c$. We compute the intrinsic dispersion from
Equation~\ref{eq:action_disp} and convert the dispersion to a $z$ offset using
the result from Figure~\ref{fig:many_orbit_wrong_ref}. We consider the result
for three different orbits with varying vertical actions of $J_z = 0.7, 7.0,
22.9\,\actunit$ ($z_{\text{max}} = 122,425,854\,\pc$) --- colored by
\Gus{blue, orange, red}, respectively. Nearby open clusters have actions
closest to the $J_z=0.7\,\actunit$ (blue) orbit, see Table~\Gus{x}.}
\label{fig:cluster_offset}
\end{figure}

In Figure~\ref{fig:cluster_offset}, we show the $z$ offset necessary to induce
the action space dispersion computed from Equation~\ref{eq:action_disp}, using
the \texttt{gala} computed mass enclosed within a radius of $8.2\,\kpc$ ($\sim
10^{11}\,M_{\odot}$).

We now convert the $z$~offset-$m_c$ relation to an $R_{\n}$-$m_c$ relation,
where $R_{\n}$ indicates a sort of ``solar neighborhood'' radius. An observer
attempting to reconstruct an open cluster of mass $m_c$ with perfect knowledge
of the midplane at the location of the cluster would expect to find the
ejected members using a dynamical cut for field stars within a distance
$R_{\n}$ of the cluster. If such an observer wants to find ejected stars that
are at a distance greater than $R_{\n}$, a more sophisticated analysis would
be required, the details of which are not apparent to the authors of this
work.

We use the relation in Figure~\ref{fig:range_deltaphi}, which shows the
typical $z$ offset at each $\Delta \phi$, to convert a given $z$ offset into a
$\Delta\phi$, which we then convert into a chord length which we interpret as
$R_{\n}$, the neighborhood around the cluster. If the $z$~offset is greater
than the maximum range value of the galaxy, we simply report the maximum chord
length ($2\times8.2\,\kpc$). The result of this procedure is shown in
Figure~\ref{fig:Rn_mc}. The solid lines indicate the neighborhood values
assuming the mean relation between $z$~offset and $m_c$, and the dashed lines
show the $+1\,\sigma$ and $-1\,\sigma$ relations, indicative of the strong
$\phi$-dependence of the midplane effect.

Figure~\ref{fig:Rn_mc} is most informative for the low $J_z$ orbit (blue). The
neighborhood around $\sim100\Msun$ clusters can be as small as $\sim300\,\pc$
but up to $\sim1.2\,\kpc$ for m12i and m12f, though the situation is
marginally better for m12m with values ranging from $\sim0.8\,\kpc$ to
$\sim3\,\kpc$. The neighborhood values steadily increase with $m_c$, topping
out at $\sim5\,\kpc$ for $\sim10^4\,\Msun$ clusters in m12i and m12f. In m12m,
the neighborhood value reaches the maximum allowed of $16.4\,\kpc$ for
clusters of mass $\sim800\,\Msun$. The interpretation of this is that the
expected action space dispersion is so high for such high mass clusters that
the measured $z$~offset is not large enough throughout the entire galaxy to be
important.

For the medium $J_z$ orbit (orange), the situation is better, although such an
orbit is not a good fit to any of the nearby open clusters \Gus{Table x}. For
m12i and m12f, the neighborhood is $\sim \text{few}\,\kpc$ for
$m_c\sim10^2\,\Msun$. At $\sim10^3\,\Msun$ the curves reach their maximum
value. For m12m, the neighborhood starts at $\sim10\,\kpc$ and quickly rises
to the maximum value. For the high $J_z$ orbit (red), the neighborhood is at
the maximum value for nearly all masses, except at the very low-mass end in
m12i.

\begin{figure*}
\plotone{fig/Rn_vs_mc.pdf}
\caption{The neighborhood around an open cluster of mass $m_c$ for our three
orbits (blue, orange, and red) and for each of the FIRE galaxies considered
here --- m12i, m12f, and m12m ({\em left}, {\em center}, and {\em right}
panels, respectively). Given the $z$ offset for a cluster of mass $m_c$
computed in Figure~\ref{fig:cluster_offset}, we convert this to a chord length
using the result from Figure~\ref{fig:range_deltaphi} and plot as a solid blue
line. We interpret this chord length as the ``neighborhood'' around the
cluster, i.e. the distance from the cluster one could go before the $J_z$
error induced by our midplane effect is comparable to the intrinsic action
dispersion of the cluster. When the $z$ offset is larger than the maximum
range in Figure~\ref{fig:range_deltaphi}, we report the maximum chord length
($2\times8.2\,\kpc$). We aso report the same procedure but using the
$\pm1\sigma$ lines from Figure~\ref{fig:range_deltaphi}, which we plot here as
dashed lines. For the medium- and high-$J_z$ orbits (orange and red), the
neighborhood is quite large and thus the effect is negligible. However, for
the low-$J_z$ orbit (blue), which is closest to the orbits of nearby open
clusters (see Table~\Gus{x}), the neighborhood is only a few hundred $\pc$ for
low-mass clusters.}
\label{fig:Rn_mc}
\end{figure*}

\appendix \section{Appendix 1}
Gonna write an appendix probably 

\bibliography{references}

\end{document}
