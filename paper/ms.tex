\documentclass[twocolumn]{aastex62}

% \submitjournal{ApJ}

\shortauthors{Beane et al.}

\begin{document}

\title{Actions are Bad, Mmkay}

\correspondingauthor{Angus Beane}
\email{abeane@sas.upenn.edu}

\author{Angus Beane}
\affil{Center for Computational Astrophysics, Flatiron Institute,
162 5th Ave., New York, NY, 10010, USA}
\affil{Department of Physics \& Astronomy, University of Pennsylvania,
209 South 33rd St., Philadelphia, PA 19104, USA}

\author{others}

\begin{abstract}

In an axisymmetric potential the 6D phase space of stars' positions and velocities can be reduced to a 3D space of invariant actions. For this reason actions are commonly used to help interpret phase space. However, it is not expected that actions are invariant in detail for the Milky Way --- it's spiral arms, bar(s), and gas clouds are all non-axisymmetric features that should induce long-term evolution in action space. However, these same features will cause a star's true orbit to deviate slightly from its orbit in an axisymmetric potential. If an observer assumes an axisymmetric potential when computing actions, this will induce an artificial short-term phase dependence of the actions, which we interpret as systematic error. Using the potential of cosmological zoom-in simulations of Milky Way Mass galaxies from the FIRE collaboration (the Latte suite), we investigate the magnitude of this systematic error. We find XXX. Something something lower limit.

\end{abstract}

\keywords{actions}

\section{Introduction} \label{sec:intro}


% \bibliography{references}

\end{document}
