\documentclass[twocolumn]{aastex62}

% \submitjournal{ApJ}

\shortauthors{Beane et al.}
\shorttitle{Our Galactic Midplane}

\usepackage{graphicx}
\usepackage{gensymb}
\usepackage{bm}

\newcommand{\Gus}[1]{\textcolor{red}{#1}}

\newcommand{\Msun}{\text{M}_\odot}
\newcommand{\pc}{\text{pc}}
\newcommand{\kpc}{\text{kpc}}
\newcommand{\Myr}{\text{Myr}}
\newcommand{\Gyr}{\text{Gyr}}
\newcommand{\kms}{\text{km}/\text{s}}
\newcommand{\actunit}{\text{kpc}\,\kms}

\newcommand{\abs}[1]{\left| #1 \right|}
\newcommand{\z}{z_r}
\newcommand{\uth}{\textsuperscript{th}}
\newcommand{\n}{\text{n}}

\newcommand{\beq}{\begin{equation}}
\newcommand{\eeq}{\end{equation}}

% affiliations
\newcommand{\cca}{Center for Computational Astrophysics, Flatiron Institute,
162 5th Ave., New York, NY 10010, USA}
\newcommand{\penn}{Department of Physics \& Astronomy, University of
Pennsylvania, 209 South 33rd St., Philadelphia, PA 19104, USA}
\newcommand{\amnh}{Department of Astrophysics, American Museum of Natural
History, Central Park West at 79th St., New York, NY 10024, USA}
\newcommand{\columbia}{Department of Astronomy, Columbia University, 550 W
120th St., New York, NY 10027, USA}

\begin{document}

\title{Our Galactic Midplane Is Probably Wrong, and Not by a Little}

% \correspondingauthor{Angus Beane}
\email{abeane@sas.upenn.edu}

\author{Angus Beane}
\affil{\cca}
\affil{\penn}

\author{Robyn Sanderson}
\affil{\cca}
\affil{\penn}

\author{Mordecai-Mark Mac Low}
\affil{\cca}
\affil{\amnh}

\author{Melissa K. Ness}
\affil{\cca}
\affil{\columbia}

\author{Daniel Angl\'es-Alc\'azar}
\affil{\cca}

\author{Megan Bedell}
\affil{\cca}

\begin{abstract}

In a completely axisymmetric potential, the notion of a global Galactic
midplane is well defined. However, since the Milky Way is not completely
axisymmetric, the global Galactic midplane is to some level of precision not
well defined. The local Galactic midplane has been found by determining the
height at which the number counts of stars reaches a maximum. This local
midplane is then used as the global midplane, an assumption which would be
justified in a completely axisymmetric potential. Assuming a global Galactic
midplane exists, we show in three cosmological zoom-in simulations from the
FIRE collaboration that local determinations of the midplane can vary by up to
a thin disk scale height. The effect is more pronounced in galaxies with less
quiescent histories (e.g. Sagittarius-like encounters). We explore one
consequence of such an inaccuracy of the midplane in the context of orbital
actions. Using standard orbit integration techniques, we show that midplane
inaccuracy induces artifical time evolution in the vertical action that makes
it virtually unusable for individual stars in the disk. The situation is less
dire for angular momentum and radial action determinations, though we also
show that inaccuracies in the solar radius can have a similar effect on the
latter. As a result, we question the use of actions as dynamical invariants in
the Milky Way disk.

\end{abstract}

\keywords{actions}

\section{Introduction} \label{sec:intro}
Our understanding of the Milky Way is currently undergoing a revolution in the
era of {\em Gaia} DR2. Recent major discoveries include the remnants of a
major merger \citep{2018Natur.563...85H}, a phase-space spiral (\Gus{cites})
indicating local substructure infall (\Gus{cites}), a gap indicative of a dark
matter subtructure in GD1 \citep{2018arXiv181103631B}, and \Gus{others}.

Underlying all of these discoveries is the assumption of a Galactocentric
coordinate system \citep{2008gady.book.....B}. In angular
coordinates, the center of the galaxy is taken to be the location of
Sgr~A\textsuperscript{*} \citep[e.g.][]{2004ApJ...616..872R}. From stellar
motions near Sgr~A\textsuperscript{*}, the distance to the center is taken to
be $\sim 8.1\text{--}8.3\,\kpc$
\citep{2009ApJ...692.1075G,2018A&A...615L..15G}. The vertical height is found
by identifying where the stellar positions and velocities reach a maximum
(effectively the median height of all disk stars), and is usually taken to be
$\sim 25\,\pc$ \citep{2001ApJ...553..184C}, with more recent measurements from
{\em Gaia} DR2 placing it at $\sim 20\,\pc$ \citep{2019MNRAS.482.1417B}.

These measurements of the galactocentric coordinate system rely on the
assumption of an axisymmetric Milky Way. Under this assumption, these
estimates would provide the correct parameters for the Galactocentric
coordinate system up to measurement error. However, the Milky Way is not
axisymmetric in detail. Spiral arms, bar(s), and infalling satellite galaxies
such as Sagittarius and the Large Magellanic Cloud all induce
non-axisymmetries.

Since the Milky Way is not truly axisymmetric, it is not clear the best way to
define the Galactocentric coordinate system --- or, perhaps, the best way to
modify the current implicit definition of the Galactocentric coordinate system
in order to account for non-axisymmetries.

In a slowly-evolving axisymmetric potential, the actions of stars' orbits are
conserved quantities \citep{2008gady.book.....B,2014RvMP...86....1S}. In other
words, if we could measure the actions of a star in such a potential, we would
measure the same actions over the course of the orbit. The non-axisymmetries
of a more realistic potential induce time evolution in the {\em measured}
actions --- the true orbital invariants (if they exist) are invariant by
definition. Through simple tests, we will see that even in a truly
axisymmetric potential, if the wrong coordinate system is assumed then there
will be additional time evolution in the measured actions.

The transformation to action space is not trivial, and many of the assumptions
of the transformation are potentially incorrect. It is therefore natural to
consider the reason to transform to action space. One reason is dimensionality
reduction --- in the action space paradigm, a stellar orbit is completely
described by three actions, as opposed to six dimensions of phase space. This
means that the relationship between {\em orbital} properties of individual
stars and stellar properties such as age or metallicity can be investigated.

To provide one such realization of a realistic Galactic potential, we turn to
cosmological zoom-ins of Milky Way-mass galaxies from the Feedback in
Realistic Environments (FIRE)
collaboration\footnote{\url{https://fire.northwestern.edu}} (the Latte suite
of simulations).

\section{Impact on Actions} \label{sec:ref_frame}
We first demonstrate the importance of measuring the Galactic midplane to very
high accuracy. We explore the consequence of an incorrect midplane in the
context of computing orbital actions of disk stars. An inaccuracy in the
midplane is only one potential systematic error in the way actions are
presently computed. The assumed axisymmetric model of the Galaxy may be a poor
description of the Galactic potential, and the parameters in this axisymmetric
model may be estimated incorrectly.

In this work, we assume both that an axisymmetric model of the Galaxy is a
good description of the potential and that we have correctly determined the
parameters of that model. We are only concerned with an inaccuracy

\subsection{Summary of Action Computation} \label{ssec:action_comp}
We begin by summarizing one way in which actions are computed in the
literature \Gus{cite}. One first assumes an axisymmetric model of the Galaxy.
This model may itself be a poor description of the Galactic potential,

We use the code \texttt{gala} v0.3 to perform orbit integrations and
conversion to action space \citep{2017JOSS....2..388P,Price-Whelan:2018},
which uses an action estimator technique presented by
\citet{2014MNRAS.441.3284S}. Another common alternative is \texttt{galpy}
\citep{2015ApJS..216...29B} using either direct orbit integration or the
\Gus{Stackel Fudge} approximation. We use the default \texttt{MWPotential} as
our potential, which is based on the Milky Way potential available in
\texttt{galpy} \citep{2015ApJS..216...29B}. This potential includes a
Hernquist bulge and nucleus \citep{1990ApJ...356..359H}, a Miyamoto-Nagai disk
\citep{1975PASJ...27..533M}, and an NFW halo \citep{1997ApJ...490..493N}, and
is fit to empirically match some observations. We use the Dormand-Prince
8(5,3) integration scheme \citep{Dormand80:integrator}. We use a timestep of
$1\,\Myr$ and integrate for $5\,\Gyr$, corresponding to $\sim 20$ orbits for a
Sun-like star.

\subsection{Cartoon} \label{ssec:cartoon}
We present a cartoon in Figure~\ref{fig:cartoon} to show that an inaccurate
determination of the midplane can lead to a phase-dependence of the {\em
observed} actions. The solid gray line indicates a hypothetical ``true''
orbit. The y-axis corresponds to the vertical height of the orbit and the
x-axis orbital phase. The dashed gray line is an incorrect midplane which an
observer uses to compute orbits.

Now suppose this observer makes a measurement of the true orbit at the orange
point or the red point. Then the orange and red lines correspond to the orbits
that the observer would compute for each point. In action space, this would
correspond to a different value of $J_z$ for the orange and red points. In
this way, assuming the wrong coordinate system induces time dependence in the
actions {\em computed by an observer}.

\begin{figure*}
\plotone{fig/cartoon.pdf}
\caption{Cartoon showing the effect an incorrect midplane can have on orbit
integrations and as an extension on action computations. The x-axis shows the
orbital phase and the y-axis the vertical height. The top gray curve depicts a
hypothetical true orbit oscillating about the true midplane, the horizontal
solid gray line. Consider an immortal observer who erroneously assumes the
midplane is located at the horizontal dashed line. Suppose this immortal
observer measures the orbit at the blue and pink points. If the immortal
observer were to perform orbit integrations at each observation, they would
assume the blue and pink curves for the respective orbits. In this way, an
incorrect midplane will induce phase-dependence in the {\em observed}
actions.}
\label{fig:cartoon}
\end{figure*}

\subsection{Quantification} \label{ssec:quant}
We now turn to classical orbital integration in a commonly assumed Milky Way
potential to quantify the time variation of computed actions for different
coordinate system errors. Our action computation procedure is described in
Section~\ref{ssec:action_comp}.

First, we illustrate the time-dependence of one possible inaccurate reference
frame. We begin by taking an orbit with initial position $(8, 0, 0)\,\kpc$ and
initial velocity $(0, -190, 50)\,\kms$ --- this corresponds to an orbit with
actions $(J_r, L_z, J_z) = (32.5, -1520, 23.0)\,\actunit$ and
$z_{\text{max}}=854\,\pc$. The orbit is plotted in \Gus{Appendix x}. We then
suppose that an observer makes observations of this orbit in a coordinate
system that is incorrect in radius by $200\,\pc$ and height by $100\,\pc$ ---
we subtract the vector $(200, 0, 100)\,\pc$ from each position in the orbit.
This corresponds to an observer physically located at the position $(8, 0,
0)\,\kpc$ but erroneously thinking they are located at $(8.2, 0, 0.1)\,\kpc$.

Using the observations of the orbit that this erroneous observer makes every
$\Myr$\footnote{In addition to being erroneous, we also assume our fictitious
observer is immortal.}, we perform the same orbit integration procedure as
before to compute the actions our observer would make in the incorrect
coordinate system over the duration of the orbit. The erroneous computed
actions are shown for the first $\Gyr$ of the orbit in
Figure~\ref{fig:one_orbit_wrong_ref}. Occasionally the numerical scheme fails
and very large actions are reported by \texttt{gala}
--- we perform a $5\sigma$ clip on each action to exclude such orbits, but
this only excludes a total of $5$ orbits out of the $1000$ considered for
Figure~\ref{fig:one_orbit_wrong_ref}. Some numerical artifacts remain, but the
vast majority of orbits are computed properly.

Figure~\ref{fig:one_orbit_wrong_ref} shows that the erroneous observer would
infer significant time evolution in their computed actions. The
95\textsuperscript{th} minus 5\textsuperscript{th} percentiles are $8.7$,
$47.1$, and $19.3\,\actunit$ for $J_r$, $L_z$, and $J_z$, respectively. These
are fractional errors of $26.5\%$, $3.1\%$, and $81.4\%$, respectively.

\begin{figure*}
\plotone{fig/one_orbit.pdf}
\caption{The artificial phase-dependence in the observed actions induced by an
error in the Galactocentric coordinate system. We consider here an orbit with
initial position $(8,0,0)\,\kpc$ and initial velocity $(0,-190,50)\,\kms$.
This corresponds to an orbit with actions $(J_r, L_z, J_z) = (32.5, -1520,
23.0)\,\actunit$ and $z_{\text{max}}=854\,\pc$. Considering the first $\Gyr$
of this orbit, we assume an immortal observer }
\label{fig:one_orbit_wrong_ref}
\end{figure*}

We now repeat the procedure but continuously changing the error in the assumed
midplane. In Figure~\ref{fig:many_orbit_wrong_ref}, we report the 95\uth minus
5\uth percentile divided by the true action value for three different orbits.
The red orbit is exactly the same as in Figure~\ref{fig:one_orbit_wrong_ref},
and we also consider two more planar orbits with initial vertical velocities
of $30\,\kms$ (\Gus{orange}) and $10\,\kms$ (\Gus{blue}). For these two new
orbits, the true actions are $(J_r, L_z, J_z) = (37.9, -1520, 7.0)\,\actunit$
and $(40.4, -1520, 0.7)\,\actunit$ for the \Gus{orange} and \Gus{blue} orbits,
respectively. The value of $z_{\text{max}}$ for these orbits are $122\,\pc$
and $425\,\pc$, respectively (with the blue orbit having $z_{\text{max}}$ of
$854\,\pc$.) \Gus{Easy way to explain change in $J_r$?}

The top row of Figure~\ref{Fig:many_orbit_wrong_ref} shows the error in each
action assuming an offset in the midplane. In the {\em bottom row} we also
consider offsets in the Galactocentric radius $R$. The {\em left}, {\em
middle}, and {\em right} columns show the 95\uth minus 5\uth percentile of the
measured action over the course of the incorrect orbit divided by the true
action value as a percentage. 

In the {\em upper middle} panel, there is essentially no error in the
determination of $L_z$. This is expected since, in principle, $L_z$ only
depends on the $x$- and $y$-components of the position and velocity of the
stars\footnote{Though this is not how we compute $L_z$ in practice.}, and is
thus unaffected by offsets in $z$.

The {\em upper right} panel shows that the fractional error in $J_z$ is more
exaggerated for more planar orbits. For the nearly planar (blue) orbit,
deviations that are $\sim30\,\pc$ already give $100\%$ deviations in the
actions. The required offset for $100\%$ deviation is $\sim120$ and
$\sim225\,\pc$ for the orange and red orbits, respectively. \Gus{Make these
numbers exact.}

For completeness, we also consider the effect of offsets in the Galactocentric
radius $R$ in the {\em bottom panels}. The error is largest for $J_r$, with
some deviations resulting in $L_z$ and relatively small deviations in $J_z$.
In the {\em bottom left} panel the orange and blue lines overlap, and in the
{\em bottom middle} and {\em bottom right} panels all three lines overlap.

\begin{figure*}
\plotone{fig/many_orbits_schmactions.pdf}
\caption{Caption.}
\label{fig:many_orbit_wrong_ref}
\end{figure*}

\section{Local Midplane in FIRE} \label{sec:local_fire}
\subsection{Description of Simulation} \label{ssec:cosmozoom}
Here we briefly describe the cosmological zoom-ins used in this work.
Cosmological zoom-ins allow one to simulate a selected region at high
resolution embedded in a low-resolution cosmological background
\citep[e.g.][]{1993ApJ...412..455K,2014MNRAS.437.1894O}. The FIRE
collaboration has simulated a number of Milky Way-mass zoom-ins as part of the
{\em Latte} suite of FIRE-2 simulations
\citep{2016ApJ...827L..23W,2018MNRAS.481.4133G}. We use the three zoom-ins
considered in \citet{2018arXiv180610564S} --- m12i, m12f, m12m, since they
have associated mock {\em Gaia} DR2 catalogues and the $z=0$ snapshots are
publicly available.

First, a dark matter only simulation was run with $\Lambda$CDM cosmology
parameterized by $\Omega_m = 0.272$, $\Omega_{\Lambda} = 0.728$, $\Omega_b =
0.0455$, $h = 0.702$, $\sigma_8 = 0.807$, and $n_s = 0.961$, consistent with
current constraints \citep{2018arXiv180706209P}. Halos are selected at $\z=0$
based solely on their mass ($\sim 1\text{--}2 \times 10^{12} \Msun$). In this
work, to avoid confusion with the vertical height $z$, we refer to
cosmological redshift as $\z$. Particles within $5 R_{200\text{m}}$ of each
halo are traced back to $\z=99$ and the initial conditions are regenerated at
higher resolution using MUSIC \citep{2011MNRAS.415.2101H}.

Simulations were run using
\texttt{GIZMO}\footnote{\url{http://www.tapir.caltech.edu/~phopkins/Site/GIZMO.html}}
\citep{2015MNRAS.450...53H}. Hydrodynamics is solved using the meshless
finite-mass method and gravity is solved using a modified version of the
Tree-PM solver of \texttt{GADGET-3}, using fully adaptive and fully
conservative gravitational force softening for gas.

These halos are uncontaminated and contain gas particles of mass $\sim 7000
\text{--} 20,000\,\Msun$ and star particles of mass $\sim 5000 \text{--} 7000\,
\Msun$\footnote{In fact, the simulation m12i contained a gas
splitting bug which causes a small number of gas and star particles to have
larger than typical masses. These constitute only $\sim0.2\%$ of the star
particles in the $z=0$ snapshots, and are only a factor of a few more massive
than typical star particles. Since our results are broadly consistent between
the three simulations, we ignore this minor complication.}, with the lower end
coming from stellar evolution \citep{2018arXiv180610564S}. Softening lengths
for dark matter and star particles are fixed at $112\,\pc$ and $11.2\,\pc$,
respectively.\footnote{This is $2.8$ times the often-quoted
Plummer-equivalent.} The gas softening length is adaptive, but at $z=0$ the
median softening length for gas particles around roughly solar positions (with
cylindrical radii within $500\,\pc$ of $8.2\,\kpc$ and $\abs{z}<1\,\kpc$) is
$98\,\pc$.

In this work, we take the coordinate systems used in
\citet{2018arXiv180610564S} as our fiducial coordinate systems for each
galaxy. The center of the galaxy is found through an iterative ``shrinking
spheres'' method, where the center is calculated for a given radius which is
then reduced by $50\%$ until converging on the galaxy's center. The center of
mass velocity is then determined by all star particles within $15\,\kpc$ of
this center. The galaxy is then rotated onto the principal axis frame as
determined by stars younger than $1\,\Gyr$ inside of the fiducial solar radius
$R_{\odot} = 8.2\,\kpc$.

% We restrict ourselves to the simulation m12i \citep[first introduced
% in][]{2016ApJ...827L..23W}. For this work, this galaxy is the best-case
% scenario. It does not contain a bar (like \texttt{m12m}) or companion galaxy,
% and by eye is the closest to axisymmetric of the three galaxies presented in
% \citet{2018arXiv180610564S} \citep{2018MNRAS.481.4133G}.

It is important to remember that simulations of Milky Way-mass galaxies are
not perfect representations of the true Milky Way, as discussed in
\citet{2018arXiv180610564S}. For instance, the velocity structure of m12i is
closer to M31 than the Milky Way. However, in this work we are most interested
in the global properties of the m12i potential, and specifically in deviations
from axisymmetry. From this perspective, m12i is actually far more
axisymmetric than we might expect of the Milky Way. While m12i has prominent
spiral arms, it lacks a bar and a companion like the Large Magellanic Cloud.
In future work, these additional complications also need to be addressed.

\subsection{Local Midplane} \label{sec:local_midplane}
We now turn to defining the local midplane that an observer might measure if
they were situated in each of these galaxies. We place our imaginary observer
at the solar radius of $8.2\,\kpc$ and azimuth $\phi$\footnote{Our erroneous,
immortal observer is also warp-drive capable.}, and consider stars within a
cylinder of radius $0.5\,\kpc$ and height $1\,\kpc$ perpendicular to the
fiducial disk.

This median height is taken to be what our observer would measure as the local
Galactic midplane at each $\phi$, and is reported for each galaxy in
Figure~\ref{fig:midplane}. We find that only $10$ iterations of this procedure
are necessary for convergence. We bootstrap resample $1000$ times and report
the $95^{\text{th}}$ minus $5^{\text{th}}$ percentile as the dashed-line error
bars. This bootstrap resampling is performed by first selecting all stars
within a height of $2\,\kpc$, resampling that selection, and then repeating
the $10$ iterations.

To allow for potential small inaccuracies in the fiducial coordinate systems,
we also subtract the best fit $\text{A} \sin{\left(\phi + \text{B}\right)} +
\text{C}$ curve from the midplane as a function of azimuth. For $\text{A}$,
the values are $-165$, $45$, $9\,\pc$ and for $\text{C}$ the values are $-69$,
$19$, and $-18\,\pc$ for m12i, m12f, and m12m, respectively \Gus{Double check
numbers}. For the assumed solar radius of $8.2\,\kpc$, we can approximate the
angle offset for the $z$-axis from the values of $\text{A}$ --- we compute
\Gus{$1.15$, $0.31$, and $0.06\,\deg$} for m12i, m12f, and m12m. These angle
offsets are consistent with the values given in \citet{2018arXiv180610564S}
for the difference between the $z$-axis as defined by the gas and stars.

Figure~\ref{fig:midplane} shows the inferred midplane our imaginary observer
would make as a function of azimuth for each galaxy. The $90\%$ interquartile
range for each galaxy is $185$, $162$, $84\,\pc$ for m12i, m12f, and m12m. 

\begin{figure*}
\plotone{fig/midplane_fit.pdf}
\caption{Caption.}
\label{fig:midplane}
\end{figure*}

\section{Discussion} \label{sec:discussion}
We have shown that, in simulation, the ``local midplane'' can have strong
azimuthal dependence. The implications of this finding are not clear, but
clearly demonstrate that assuming our local midplane is the same as the global
midplane is insufficient for extracting the most dynamical information from
the {\em Gaia} mission.

To gain further handle on the midplane behavior as a function of azimuth, we
compute the range of midplane values for a given $\Delta \phi$ segment of the
midplane in Figure~\ref{fig:range_deltaphi}. With the observer's azimuth set
to zero, we consider a region extending from $-\Delta \phi/2$ to
$\Delta\phi/2$. We also compute the chord length for the chord extending from
the observer to $\pm \Delta\phi/2$. We repeat this for each of the $50$ bins
in azimuth. Each bin is plotted as a gray line, with the mean value in solid
blue and the $1\,\sigma$ dispersion as the dashed blue lines.

\begin{figure*}
\plotone{fig/range_dphi.pdf}
\caption{Caption.}
\label{fig:range_deltaphi}
\end{figure*}

A possible interpretation of our finding is that one should use a star's local
midplane in order to compute dynamical quantities like the vertical action.
Under the assumption that this interpretation is correct, we now consider the
implications for dynamical studies of disrupting open clusters (the original
motivation of this study).

Suppose an open cluster of mass $M_c$ on an orbit with actions $\bm{J}_c$ is
disrupted, and that the cluster is currently at an orbit phase of
$\bm{\theta}_0$. We would like to determine all stars in the {\em Gaia}
dataset\footnote{This is likely to be more successful with DR3/4 than DR2,
since upcoming data releases will contain a far greater number of stars with
radial velocities --- though targeted, follow-up radial velocity surveys may
allow this program to be performed sooner.} that are likely to be ejected
members from that cluster. One could proceed by integrating the cluster's
orbit and selecting all stars within a certain distance of that orbit,
accounting for the fact that the stream will not exactly follow the orbit of
the cluster \citep[e.g.][]{2011MNRAS.413.1852E}.

One could then make a further selection by requiring that the action of each
field star $i$ be close to the cluster's actions to within some bound, i.e.
$\bm{J}_i - \bm{J}_c \in V_e$ where $V_e$ is some volume in action space
enclosing a region of acceptable error. A natural choice for $V_e$ is to
center it at the origin and allow its extent to be the intrinsic action-space
dispersion of a disrupted cluster. This dispersion can be estimated as
\citep[\S~8.3.3][]{2008gady.book.....B}
\beq \label{eq:action_disp}
\frac{\Delta J_i}{J_i} \sim \left(\frac{m_c}{M}\right)^{1/3}\text{,}
\eeq
where $m_c$ is the mass of the cluster and $M$ is the mass enclosed by the
cluster's orbit. A more sophisticated treatment of the action-space dispersion
of disrupted clusters can be obtained by following
\citet{2011MNRAS.413.1852E}, but such a treatment is premature for this work.

The act of comparing a field star's action to a cluster's action (i.e. in
computing $\bm{J}_i - \bm{J}_c$) is subject to significant error if the
cluster and field star have significantly different local midplanes and an
observer assumed a single global midplane.

To explore the magnitude of this effect, we consider the following program.
First, we assume a cluster mass $m$ and estimate the expected action-space
dispersion from Equation~\ref{eq:action_disp}. We then consult
Figure~\ref{fig:many_orbit_wrong_ref} to determine the $z$ offset necessary in
order for the induced error in the vertical action to be the same as the
expected action-space dispersion. If this were the case, an observer would
have to account for the different midplane of the field star and the cluster.
We repeat this analysis for each of the three orbits considered in
Section~\ref{ssec:quant}.

With a characteristic $z$ offset in hand, we consult
Figure~\ref{fig:range_deltaphi} to convert this $z$ offset into a $\Delta
\phi$, which we convert to an arclength or characteristic distance (the
difference between the two will be negligible for the small $\Delta \phi$'s we
will compute). Therefore, if an observer is attempting to reconstruct a
cluster with a given mass on a given orbit, we will give a characteristic
distance between a field star and the cluster within which an observer might
not need to consider differences in their respective local midplanes.

\begin{figure}
\plotone{fig/cluster_offset.pdf}
\caption{Caption.}
\label{fig:cluster_offset}
\end{figure}

In Figure~\ref{fig:cluster_offset}, we show the $z$ offset necessary to induce
the action space dispersion computed from Equation~\ref{eq:action_disp}, using
the \texttt{gala} computed mass enclosed within a radius of $8.2\,\kpc$ ($\sim
10^{11}\,M_{\odot}$).

We now convert the $z$~offset-$m_c$ relation to an $R_{\n}$-$m_c$ relation,
where $R_{\n}$ indicates a sort of ``solar neighborhood'' radius. An observer
attempting to reconstruct an open cluster of mass $m_c$ with perfect knowledge
of the midplane at the location of the cluster would expect to find the
ejected members using a dynamical cut for field stars within a distance
$R_{\n}$ of the cluster. If such an observer wants to find ejected stars that
are at a distance greater than $R_{\n}$, a more sophisticated analysis would
be required, the details of which are not apparent to the authors of this
work.

We use the relation in Figure~\ref{fig:range_deltaphi}, which shows the
typical $z$ offset at each $\Delta \phi$, to convert a given $z$ offset into a
$\Delta\phi$, which we then convert into a chord length which we interpret as
$R_{\n}$, the neighborhood around the cluster. If the $z$~offset is greater
than the maximum range value of the galaxy, we simply report the maximum chord
length ($2\times8.2\,\kpc$). The result of this procedure is shown in
Figure~\ref{fig:Rn_mc}. The solid lines indicate the neighborhood values
assuming the mean relation between $z$~offset and $m_c$, and the dashed lines
show the $+1\,\sigma$ and $-1\,\sigma$ relations, indicative of the strong
$\phi$-dependence of the midplane effect.

Figure~\ref{fig:Rn_mc} is most informative for the low $J_z$ orbit (blue). The
neighborhood around $\sim100\Msun$ clusters can be as small as $\sim300\,\pc$
but up to $\sim1.2\,\kpc$ for m12i and m12f, though the situation is
marginally better for m12m with values ranging from $\sim0.8\,\kpc$ to
$\sim3\,\kpc$. The neighborhood values steadily increase with $m_c$, topping
out at $\sim5\,\kpc$ for $\sim10^4\,\Msun$ clusters in m12i and m12f. In m12m,
the neighborhood value reaches the maximum allowed of $16.4\,\kpc$ for
clusters of mass $\sim800\,\Msun$. The interpretation of this is that the
expected action space dispersion is so high for such high mass clusters that
the measured $z$~offset is not large enough throughout the entire galaxy to be
important.

For the medium $J_z$ orbit (orange), the situation is better, although such an
orbit is not a good fit to any of the nearby open clusters \Gus{Table x}. For
m12i and m12f, the neighborhood is $\sim \text{few}\,\kpc$ for
$m_c\sim10^2\,\Msun$. At $\sim10^3\,\Msun$ the curves reach their maximum
value. For m12m, the neighborhood starts at $\sim10\,\kpc$ and quickly rises
to the maximum value. For the high $J_z$ orbit (red), the neighborhood is at
the maximum value for nearly all masses, except at the very low-mass end in
m12i.

\begin{figure*}
\plotone{fig/Rn_vs_mc.pdf}
\caption{Caption.}
\label{fig:Rn_mc}
\end{figure*}

% {\em Gaia} DR2 has ushered in a new era of Galactic dynamics. High-precision
% phase space measurements have been made for an unprecedented number of stars.
% In the face of such large data, it is worth considering the goal of such
% measurements --- a goal which should inform the usage of such data.

% In this work we considered one such question: given the current position in
% phase space of a single star, what can we infer about that star's past phase
% space trajectory? Current methodology involves using a smoothed, analytic form
% for the Galactic potential (with perhaps crude implementations of spiral arms
% or sattelites), fitting such form to some selection of observations, and
% integrating from the currently measured position in phase space. This orbit
% integration is also commonly accompanied with a transformation into
% action-angle space. Other than the integration, every step of that process
% should be viewed with significant scrutiny.

% The more general problem of measuring the orbits of stars in cosmological
% simulations is essentially unexplored in this work and the literature, and we
% delay discussion to future work.

% \subsection{Dynamical Midplane}
% We now seek to define the midplane in a dynamically consistent manner, which
% we refer to as the ``dynamical midplane''. By this, we mean we would like to
% define the midplane in a manner consistent with some sort of dynamical
% constraint. Our method, which we describe now, is just one of many possible
% ways the dynamical midplane can be defined.

% We define the dynamical midplane based on the principle that perfectly
% circular, planar orbits will trace the midplane. Once the circular, planar
% orbits have been identified, the procedure for finding the dynamical midplane
% is relatively simple. From the measurement of each star's orbit, the midplane
% of the galaxy can be computed at $z=0$ easily. Furthermore, we will find that
% each orbit also gives a measure of the: dynamical center of the galaxy,
% velocity of the galaxy, and acceleration of the galaxy.

% We pause briefly to consider general difficulties in measuring stellar orbits
% in a cosmological simulation. It may seem reasonable to place the galaxy in an
% inertial reference frame and then to simply measure the positions of each star
% particle over time. However, it is not generally the case that the center of
% mass of the stellar disk experiences a vanishing acceleration. Indeed, in a
% re-run of the simulations presented here\footnote{The re-run contained a
% cosmic ray bug fix which is unimportant here, but the re-run also output the
% acceleration of each particle, which is not the case for the simulations in
% this work.}, we measure a total acceleration of $\sim 0.4\,\kms/\text{Myr}$
% for stars with cylindrical radii within $0.5\,\kpc$ of $8.2\,\kpc$ and
% vertical heights less than $0.4\,\kpc$. Assuming a constant acceleration, over
% $22\,\Myr$ --- the time spacing of snapshots considered in this work --- this
% results in a $\sim90\,\pc$ shift of the disk. Since the offset scales with
% $t^2$, this presents a significant problem for measuring stellar orbits over
% longer timescales.

% Another option is to recenter and reorient the galaxy at each snapshot.
% However, the centering and orientation procedures are not without
% stochasticity themselves. This stochasticity can easily dominate the magnitude
% of the orbit. For instance, a shift in the direction of the $z$-axis by a mere
% $0.7\degree$ results in a shift in the midplane of $100\,\pc$ at a radius of
% $8.2\,\kpc$. We attempted this procedure in various ways (i.e. exploring
% different centering and orientation procedures), but were not able to cleanly
% measure the orbits of disk stars.

% Here we choose to simplify the problem by considering only perfectly circular,
% planar orbits. Since these orbits have no phase-dependent structure, we can
% work on shorter timescales, i.e. for orbits with significant radial or
% vertical action a large fraction of their orbit would need to be sampled in
% order to measure the orbit accurately. Because we know the analytic form of
% these orbits, we can forward model the trajectory of each star particle in the
% cosmological simulation. Working in comoving coordinates, we can write down
% the trajectory of a star particle as
% \beq\label{eq:helix}
% \bm{r}(t) = R(\theta, \phi, \psi) \left[ (A\cos{\omega t}, B\sin{\omega t}, C
% t) + \bm{x_0} \right]\text{,}
% \eeq
% where $R(\theta, \phi, \psi)$ is an Euler angle transformation, $A$, $B$, $C$,
% and $\omega$ are arbitrary constants, and $\bm{x_0}$ is an arbitrary 3-vector.
% This equation is a helix with an arbitrary constant and rotation. It describes
% the

% We minimize the summed squared difference between Equation~\ref{eq:helix} and
% the true trajectory of each star with $R$ within $1\,\kpc$ of $R_{\odot} =
% 8.2\,\kpc$ in each simulation. We consider minimum rms error (i.e. the typical
% error in the Equation~\ref{eq:helix} model in describing the true trajectory).
% This rms error characterizes how well the helix model we consider, which would
% be exactly correct if the overall motion of the galaxy is well-described by
% constant speed motion and the orbit of a particular store is exactly circular
% and planar.

% We plot a histogram of the rms error for each galaxy in
% Figure~\ref{fig:something}. As we estimated before, the total effect of the
% galaxy's acceleration is roughly $90\,\pc$ (\Gus{Check for m12f, m12m}). We
% find that this is small compared to the typical error, even for the stars
% whose trajectories are best modeled by Equation~\ref{eq:helix}.

% There are many reasons for the failure of this model. First, there could be no
% stars on circular orbits. This seems unlikely considering that typical
% distribution functions peak on perfectly circular, planar orbits
% \citep[e.g.][]{2013MNRAS.434..652T}. Second, the trajectory of the galaxy may
% not be well-described by constant-speed motion. This also seems unlikely,
% considering that the bulk acceleration of the stellar disk only results in a
% $\sim90\,\pc$ (\Gus{m12f, m12m}) deviation from constant-speed motion over the
% $22\,\Myr$ time we considered.

% Furthermore, in Figure~\ref{fig:something} we plot the center of mass of the
% galaxy (in comoving coordinates) for the $22\,\Myr$ timeframe described. The
% center of mass at each snapshot is computed using the iterative shrinking
% spheres method described in Section~\ref{ssec:cosmozoom}. The rms error on the
% constant-speed motion is $\sim\text{something}$, which we interpret as
% \Gus{something}.

% One potential explanation is that {\em no star can remain on a circular,
% planar orbit}. The order unity density variations may be strong enough such
% that each star's orbit is perturbed. If there are truly stars in the
% simulation on perfectly circular, planar orbits, then the magnitude of their
% acceleration over the last $22\,\Myr$ must be constant. Furthermore, their
% trajectory in acceleration space must be well-described by a rotation. In the re-runs, 

% \section{Optimally Axisymmetric Frame} \label{sec:oa_frame}
% An analytic, axisymmetric potential with a known coordinate system is a far
% too simple description of the Milky Way. We thus turn to cosmological zoom-in
% simulations as approximations to a ``true'' Milky Way potential. We will
% perform a few simulations of small clusters embedded in the potential of a
% frozen $\z=0$ snapshot of one such zoom-in. We take our fiducial reference
% frame from \citet{2018arXiv180610564S}.

% Freezing the potential means that during the course of our simulation the
% cluster will rapidly pass by much of the substructure in the potential. To
% illustrate that this has little effect on our results, we resimulate each
% cluster with the softening lengths of the embedded potential increased by a
% factor of \Gus{something}.

% Once we perform this simulation, we calculate the actions of each star in the
% cluster assuming the best-fit smooth, axisymmetric potential, as we do
% observationally for the Milky Way. We refer to this space as the observational
% action space. We will find that there is significant time evolution in
% observational action space. Attributing some of this evolution to our fiducial
% coordinate system being incorrect, we minimize the time-dispersion in action
% space

% \subsection{Cluster Simulations} \label{ssec:cluster_sim}
% We simulate a small star cluster of $256$ stars in the frozen $\z=0$
% potential of m12i. We use the code \texttt{ph4} within the
% \texttt{AMUSE} framework \citep{2011ascl.soft07007P, 2013CoPhC.184..456P,
% 2013A&A...557A..84P} for the cluster integration. We use a softening length
% of $0.01\,\pc$ which means we do not correctly solve the internal dynamics of
% the cluster, but this is not important for our purposes. To couple the
% gravitational forces of the frozen snapshot to the cluster stars, we use the
% \texttt{BRIDGE} \citep{2007PASJ...59.1095F}, also implemented in
% \texttt{AMUSE}. Our simulation runs for $2\,\Gyr$, allowing us to solve for
% $\sim 10$ orbits.

% To compute the forces from the frozen potential, we use the simple tree code
% \texttt{pykdgrav}\footnote{\url{https://github.com/omgspace/pykdgrav},
% commit: \texttt{fbd1ddc}}. We use an opening angle of $0.5$ and exclude stars
% at distances $R>50\,\kpc$ in the fiducial coordinate system.

% We use the initial position and velocity of two star particles in
% m12i. These two star particles are randomly chosen from all star
% particles in the simulation with the following properties (in the fiducial
% coordinate system):
% \begin{enumerate}
%     \item age in the range $(0.25, 0.75)\,\Gyr$,
%     \item cylindrical radius in the range $(7.7, 8.7)\,\kpc$,
%     \item height (absolute value) less than $0.5\,\kpc$,
%     \item $J_r$ and $J_z$ in the range:
%         \subitem $(10, 20)\,\actunit$ for the first star,
%         \subitem $(40, 60)\,\actunit$ for the second star.
% \end{enumerate}
% The ids we choose are \texttt{23693026} for the first star and
% \texttt{17012804} for the second star. We refrain from calling these ``low''
% or ``high'' action stars since the actions are computed in a presumably
% incorrect coordinate system.

% To compute actions, both in the choosing of each of these star particles and
% in what follows, we use the package
% \texttt{agama}\footnote{\url{https://github.com/GalacticDynamics-Oxford/Agama},
% commit: \texttt{0e48336}}. Using only the high-resolution particles, we
% compute the symmetrized potential in two components. First, a multipole
% expansion with $20$ radial grid points and a max order of $2$ for the
% spherical-harmonic expansion is used for the dark matter. Second, a
% cylindrical spline expansion with $20$ radial and $z$ grid points, a $0$ order
% max index azimuthal-harmonic expansion, $0.2$ and $50\,\kpc$ for the minimum
% and maximum radial grid node, $0.02$ and $10\,\kpc$ for the minimum and
% maximum $z$ grid node is used for the baryonic matter (stars and gas). In both
% cases, axisymmetry is explicitly enforced. Other \texttt{agama} parameters are
% left as default.

% \subsection{Frame Optimization} \label{ssec:frame_opt}
% Figure~\Gus{something} shows the action computed at each time step in the
% fiducial coordinate system. Clearly there is significant time evolution.

% Even though we change the center, we do not recalculate the center velocity.
% This is largely irrelevant, as the recalculated center velocity differs from
% the fiducial center velocity by $<0.1\,\kms$. In recentering the galaxy, the
% package \texttt{gizmo\_analysis} will subtract a slightly different value for
% the Hubble flow from each particle's velocity to convert to a physical
% coordinate system. The error this introduces is $\sim 0.02\,\kms$, which we
% also ignore.

\appendix \section{Appendix 1}
Gonna write an appendix probably 

\bibliography{references}

\end{document}
